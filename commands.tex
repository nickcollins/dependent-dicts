% !TEX root = main.tex

\newcommand{\mynote}[3]{\textcolor{#3}{\textsf{{#2}}}}
\newcommand{\rkc}[1]{\mynote{rkc}{#1}{blue}}
\newcommand{\nick}[1]{\mynote{cy}{#1}{purple}}

\newcommand{\cvert}{{\,{\vert}\,}}

%% https://tex.stackexchange.com/questions/9796/how-to-add-todo-notes
\newcommand{\rkcTodo}[1]{\todo[linecolor=blue,backgroundcolor=blue!25,bordercolor=blue]{#1}}

\newcommand{\mattTodo}[1]{\todo[linecolor=green,backgroundcolor=green!2,bordercolor=green]{\tiny\textit{#1}}}
\newcommand{\mattOmit}[1]{\colorbox{yellow}{(Matt omitted stuff here)}}

\def\parahead#1{\paragraph{\textbf{#1.}}}
%% \def\paraheadNoDot#1{\paragraph{{\textbf{#1}}}}
\def\subparahead#1{\paragraph{\textit{#1.}}}
%% \def\paraheadindent#1{\paragraph{}\textit{#1.}}
%% \def\paraheadindentnodot#1{\paragraph{}\textit{#1}}

\newcommand{\ie}{i.e.}
% \newcommand{\eg}{{\emph{e.g.}}}
% \newcommand{\etc}{{\emph{etc.}}}
% \newcommand{\cf}{{\emph{cf.}}}
% \newcommand{\etal}{{\emph{et al.}}}

%% \newcommand{\hazel}{\ensuremath{\textsc{Hazel}}}
%% \newcommand{\sns}{\ensuremath{\textsc{Sketch-n-Sketch}}}
%% \newcommand{\deuce}{\ensuremath{\textsc{Deuce}}}
\newcommand{\Elm}{\ensuremath{\textsf{Elm}}}
\newcommand{\sns}{\ensuremath{\textrm{Sketch-n-Sketch}}}
\newcommand{\deuce}{\ensuremath{\textrm{Deuce}}}

\newcommand{\sectionDescription}[1]{\section{#1}}
\newcommand{\subsectionDescription}[1]{\subsection{#1}}
\newcommand{\subsubsectionDescription}[1]{\subsubsection{#1}}
%% \newcommand{\subsectionDescription}[1]{\subsection*{#1}}
\newcommand{\suppMaterials}{the Supplementary Materials}

\newcommand{\defeq}{\overset{\textrm{def}}{=}}

\newcommand{\eap}{action suggestion panel\xspace}
\newcommand{\Eap}{Action suggestion panel\xspace}

\newcommand{\myfootnote}[1]{\footnote{ #1}}

\def\sectionautorefname{Section}
\def\subsectionautorefname{Section}
\def\subsubsectionautorefname{Section}

\newcommand{\code}[1]{\lstinline{#1}}
\newcommand{\str}[1]
  {``#1''}

% Make italic?
%\newcommand{\Property}[1]{\emph{#1}}
\newcommand{\Property}[1]{\textrm{#1}}

% Calling out Cyrus's favorite verb, 'to be' ;)
\newcommand{\IS}{\colorbox{red}{is}\xspace}

\newcommand{\codeSize}
  %% {\footnotesize}
  {\small}

%\newcommand{\JoinTypes}[2]{\textsf{join}~~#1~~#2}
\newcommand{\JoinTypes}[2]{\textsf{join}(#1,#2)}

%% \newtheorem{theorem}{Theorem}[section] % 2.1, 2.2, etc.
\newtheorem{theorem}{Theorem}             % 1, 2, etc.
\newtheorem{remark}[theorem]{Remark}

%%%%%%%%%%%%%%%%%%%%%%%%%%%%%%%%%%%%%%%%%%%%%%%%%%%%%%%%%%%%%%%%%%%%%%%%%%%%%%%%
%% Spacing

\newcommand{\sep}{\hspace{0.06in}}
\newcommand{\sepPremise}{\hspace{0.20in}}
\newcommand{\hsepRule}{\hspace{0.20in}}
\newcommand{\vsepRuleHeight}{0.08in}
\newcommand{\vsepRule}{\vspace{\vsepRuleHeight}}
\newcommand{\miniSepOne}{\hspace{0.01in}}
\newcommand{\miniSepTwo}{\hspace{0.02in}}
\newcommand{\miniSepThree}{\hspace{0.03in}}
\newcommand{\miniSepFour}{\hspace{0.04in}}
\newcommand{\miniSepFive}{\hspace{0.05in}}
\newcommand{\breakAndIndent}
  {\mbox{}

   %% \hspace{0.15in}
   \hspace{0.00in}
  }

\newcommand{\justIndent}
  %% {\hspace{0.15in}
  {\hspace{0.00in}
  }

%%%%%%%%%%%%%%%%%%%%%%%%%%%%%%%%%%%%%%%%%%%%%%%%%%%%%%%%%%%%%%%%%%%%%%%%%%%%%%%%

% \lstset{
% %mathescape=true,basicstyle=\fontsize{8}{9}\ttfamily,
% literate={=>}{$\Rightarrow$}2
%          {<=}{$\leq$}2
%          {->}{${\rightarrow}$}1
%          {\\\\=}{\color{red}{$\lambda$}}2
%          {\\\\}{$\lambda$}2
%          {**}{$\times$}2
%          {*.}{${\color{blue}{\texttt{*.}}}$}2
%          {+.}{${\color{blue}{\texttt{+.}}}$}2
%          {<}{${\color{green}{\lhd}}$}1
%          {>?}{${\color{green}{\rhd}}$?}2
%          {<<}{${\color{green}{\blacktriangleleft}}$}1
%          {>>?}{${\color{green}{\blacktriangleright}}$?}2
%          {\{}{${\color{blue}{\{}}$}1
%          {\}}{${\color{blue}{\}}}$}1
%          {[}{${\color{purple}{[}}$}1
%          {]}{${\color{purple}{]}}$}1
%          {(}{${\color{darkgray}{\texttt{(}}}$}1
%          {)}{${\color{darkgray}{\texttt{)}}}$}1
%          {]]}{${\color{gray}{\big(}}$}1
%          {]]}{${\color{gray}{\big)}}$}1
% }

%%%%%%%%%%%%%%%%%%%%%%%%%%%%%%%%%%%%%%%%%%%%%%%%%%%%%%%%%%%%%%%%%%%%%%%%%%%%%%%%

\newcommand{\li}[1]{\lstinline[basicstyle=\ttfamily\fontsize{9pt}{1em}\selectfont]{#1}}
\newcommand{\lismall}[1]{\lstinline[basicstyle=\ttfamily\fontsize{9pt}{1em}\selectfont]{#1}}
\newcommand{\mapP}{\{0 \rightarrow \text{"a"}, 1 \rightarrow \text{"b"}, 3 \rightarrow \text{"c"}, 6 \rightarrow \text{"d"}, 10 \rightarrow \text{"e"}\}}
\newcommand{\dictP}{[(0, "a"), (0, "b"), (1, "c"), (2, "d"), (3, "e")]}

%%%%%%%%%%%%%%%%%%%%%%%%%%%%%%%%%%%%%%%%%%%%%%%%%%%%%%%%%%%%%%%%%%%%%%%%%%%%%%%%

%% \newcommand{\sal}{simple association list}
\newcommand{\sal}{association list}
\newcommand{\Sal}{Association List}
\newcommand{\SAL}{AL}

%% \newcommand{\cal}{canonicalized association list}
\newcommand{\cal}{canonical association list}
\newcommand{\Cal}{Canonical Association List}
\newcommand{\CAL}{CAL}
\newcommand{\Cals}{Canonical association lists}

%% \newcommand{\fpf}{finite partial function}
\newcommand{\fpf}{partial function}
\newcommand{\Fpf}{Partial Function}
\newcommand{\Fpfs}{Partial functions}
\newcommand{\FPF}{PF}

%% trying this out...
%% \newcommand{\cfpf}{canon. finite partial function}
\newcommand{\fpfk}{partial function with keys}
\newcommand{\Fpfk}{Partial Function with Keys}
\newcommand{\FPFK}{PFK}

\newcommand{\dd}{delta dictionary}
\newcommand{\dds}{delta dictionaries}
\newcommand{\Dd}{Delta Dictionary}
\newcommand{\Ddls}{Delta dictionaries}
\newcommand{\DD}{DD}

%%%%%%%%%%%%%%%%%%%%%%%%%%%%%%%%%%%%%%%%%%%%%%%%%%%%%%%%%%%%%%%%%%%%%%%%%%%%%%%%

\newcommand{\SemTot}{Semantic Totality}
%% \newcommand{\SemInj}{Semantic Injectivity}
\newcommand{\SemInj}{Extensionality}
\newcommand{\EqDec}{Decidable Equality}
%% \newcommand{\EzDstr}{Ease of Destruction}
\newcommand{\EzDstr}{Easy Destructibility}

%% tweak and remove indirection macros later, once terminology settles
\newcommand{\Total}{\SemTot}
\newcommand{\Extensional}{\SemInj}
\newcommand{\DecidableEq}{\EqDec}
\newcommand{\Destructible}{\EzDstr}

\newcommand{\total}{Total}
\newcommand{\extensional}{Extensional}
\newcommand{\decidable}{Decid. Eq.}
\newcommand{\destructible}{Destructible}

%%%%%%%%%%%%%%%%%%%%%%%%%%%%%%%%%%%%%%%%%%%%%%%%%%%%%%%%%%%%%%%%%%%%%%%%%%%%%%%%

% for use in alltt
\newcommand{\altFAll}{\ensuremath{\forall}}
\newcommand{\altRArr}{\ensuremath{\rightarrow}}
\newcommand{\altRARR}{\ensuremath{\Rightarrow}}
\newcommand{\altVdash}{\ensuremath{\vdash}}
\newcommand{\altAnd}{\ensuremath{\land}}
\newcommand{\altOr}{\ensuremath{\lor}}
\newcommand{\altNE}{\ensuremath{\ne}}
\newcommand{\altEmpty}{\ensuremath{\varnothing}}
\newcommand{\altIn}{\ensuremath{\in}}
\newcommand{\altNIn}{\ensuremath{\notin}}
\newcommand{\altSum}{\ensuremath{\sum}}
\newcommand{\altLAng}{\ensuremath{\langle}}
\newcommand{\altRAng}{\ensuremath{\rangle}}
\newcommand{\altLamb}{\ensuremath{\lambda}}
\newcommand{\altGam}{\ensuremath{\Gamma}}
\newcommand{\altCirc}{\ensuremath{\circ}}
\newcommand{\altCdot}{\ensuremath{\cdot}}
