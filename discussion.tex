%% \section{Discussion and Related Work}
\section{Discussion}
\label{sec:Discussion}

We conclude with a broader discussion of the design space, as well as related
and future work.

\vspace{0.05in} %% SPACE HACK

\subsection{Design Tradeoffs}
\label{sec:Tradeoffs}

\parahead{The Difficulty of Destruction}

\SemInj{} requires canonical ordering and deduplication, and \SemTot{} means that the canonical order and deduplication have to come from
how the data is interpreted rather than how it is organized. The non-literal way in which key data is interpreted means that it is not
safe for client code to work with the raw data directly---rather, all interaction with the data must be
encapsulated in library functions.

Unfortunately, this includes \emph{destruction}; an interaction which
normally goes through the very natural, elegant, and well-supported mechanism of pattern matching is now
only available through the library function \verb+destruct+. \autoref{thm:EzDstr} proves that this function
destructs the \dd{} in essentially the same way that a case expression destructs a list of pairs (although
the order in which bindings are plucked away is arbitrary from the client's perspective).

\verb+destruct+ achieves the same purpose as typical pattern matching (albeit more awkwardly), but because it
does not harness the primitive notions of pattern matching and structure, it does not establish structural
decrease on the dictionary object, which may break out-of-the-box structural recursion in the likely case
of recursion on \verb+dd'+. To enable manual establishment of structural recursion, via an extra parameter that
explicitly tracks the dictionary length, we provide another theorem:

\begin{alltt}
  extend-size :
    \altFAll\{V\} \{dd : DD V\} \{k : K\} \{v : V\} \altRArr
      k \altNIn dd \altRArr
      || dd ,, (k , v) || = 1 + || dd ||
\end{alltt}

Although possible, manually establishing termination is painful, especially given that it is not necessary
for {\sal}s or {\cal}s. Thus \dds~ fail at \EzDstr, although they are still better than {\fpf}s in this
regard, seeing as it is not only hard but impossible to destruct {\fpf}s.

Is it possible to have a data structure that possesses all four properties?
Not likely. As mentioned above, the combination of \SemInj{} and \SemTot{} essentially
necessitates that the data in the data structure be interpreted in a non-literal way, which in turn
makes it dangerous for client code to mess with the raw data. As such, we believe we have uncovered an
inherent trade-off between desirable properties. In cases where \SemTot{} and \EqDec{} are critical,
and there is little to no need to destruct dictionaries, \dds{} seem to be a clear winner over
the conventional solutions. But in cases where \SemTot{} is not so important,
but inspection or destruction are, {\cal}s may be a preferable.

\parahead{Generality}
This scenario may seem contrived, but it is actually a simplification of a real task faced by the authors.
The task was to mechanize the formalization of Smyth~\citep{smyth}, a program synthesis technique which uses
natural semantics (\ie{} big-step environment-based evaluation) and which requires assertions.
Although this mechanization has not been completed, challenges (related to the aforementioned issues)
faced during its development necessitated the invention of \dds.
\rkc{Provide URL to GitHub commit? Say omitted due for review?}
\rkc{Do we need to anonymize the Smyth name and reference for review?}

Perhaps, instead of using \dds, these issues could be worked around, by changing or dropping some of our criteria?
In many cases, substitution can be used to avoid environments---however, environments are often preferred because
they allow the formalization to more closely resemble the implementation~(\eg{}~\citep{Ancona:2014})\rkc{Reference okay?}. Furthermore,
Smyth has hole closures in addition to lambda closures, so closing over environments would be necessary regardless.

\emph{Contraction} and \emph{exchange} are not always necessary properties for program foundation judgments---%
rather, substructural type systems, by definition, deliberately violate one or both of these properties.%
\footnote{\hspace{0.01in}Technically, some substructural type systems (\eg{} \emph{relevant} type systems) uphold \emph{contraction} and \emph{exchange} but violate \emph{weakening}.}
That said, a judgment should generally uphold the \emph{structural properties} unless there is a strong and explicit
reason not to. Proving the right properties
is both a matter of good housekeeping and necessity: they are standard
considerations because they arise very naturally in any other
interesting metatheory, and the inability to prove them is often a large red flag indicating a bug in the judgment's definition.
Because \emph{contraction} and \emph{exchange} are properties of \dds{} themselves,
the proofs of these properties for one or many judgments come ``for free''.
Additionally, the useful properties of \dds{} may greatly reduce the difficulty of the usually
more judgement-specific proof of \emph{weakening}.

It may not seem useful to assert that two functions are intensionally identical, but if assertion is relaxed,
so as to test consistency or partially extensional equality instead of purely intensional equality,
it would require environments that are destructible.
Scenarios which require destructibility but not \EqDec{} lead to the same conclusions,
seeing as these columns are identical in \autoref{fig:prop-summary}.

In addition to breaking strict positivity, refinement proofs can be the source of less severe pain points in
a broader range of circumstances.
The ability to refine ordinary types with proofs of validity is one of the most interesting and useful benefits of
dependently-typed languages, and this power should be appreciated. However, refining with proof terms can
come at a practical cost, so even in dependently-typed languages, there is high value in avoiding refinements
whenever possible (or at least whenever profitable). The practical cost of refinements is that proof terms
do not possess the properties \SemInj~ or \EqDec: due to \emph{proof relevance} (TODO source), two proofs of
the same property may be unequal, and due to the fact that proof terms may contain functions, proof terms do not
possess \EqDec~ in the general case. Thus, \SemTot---which obviates the need for refinement---has
a lot of practical value in many general cases beyond those where it is absolutely necessary to appease the
positivity checker.

\parahead{Dictionaries vs. Custom Datatypes}

Dictionaries are not always used to represent environments in mechanizations of programming language theory.

When the order of bindings does not matter, dictionaries are a natural choice for type contexts.
%
But if types defined ``later,'' or in inner scopes, can refer to types defined ``earlier,'' or in outer scopes, then order-insensitive dictionaries are inherently inappropriate.
%
This is the case for languages that support subtyping---notably, the subject of the POPLmark Challenge~\citep{POPLmark}.
%
Linear logics, on the other hand, require sensitivity to duplicate insertions, so duplication-insensitive dictionaries are inappropriate for them as well.
%
Though {\sal}s remain the most natural choice in these cases, future work could explore data structures that are sensitive to ordering but not duplication, or vice versa.

The use of dictionaries is furthermore avoided in many systems that use substitution rather than environments
%
in defining the dynamic semantics of a language, as well as the many systems that avoid named variables altogether by using De Bruijn indices \cite{XXX,XXX,XXX} or comparable techniques.

For these reasons, many existing mechanizations make little use of order-and-duplication-insensitive dictionaries. 
%
However, as mechanization becomes increasingly popular, for an ever-broadening scope of applications,
%
it seems inevitable that a programming utility as fundamental as dictionaries will eventually become ubiquitous, at which point it will be important to have the
%
best implementations at our disposal.

\parahead{Performance Concerns}
%
It was noted above that AVL implementations are apparently more popular than {\cal}s, and as more software is mechanized, performance is likely to become an
%
even greater concern than it is today. In cases where there is no extraction step, and the proof language is also the language that will be executed,
%
there may be no choice but to use implementations that are high-performance but theoretically unwieldy. However, it seems more appropriate, especially with
%
fundamental utilities such as dictionaries, to either extract or translate the mechanization into an implementation language that defines these utilities natively
%
by way of highly efficient implementations such as hashtables or red-black trees. This extraction may not be completely fidelitous, since it will be using a
%
different data structure in the implementation than was used in the proofs, but presumably the implementation's version of dictionaries is well-tested and bug-free,
%
and its definition of equality is, at least for fundamental utilities such as dictionaries, extensional and decidable. Regardless, even if unperformant implementations
%
cannot be used for code that will be run, they may be useful for parts of the code which are only used for proofs and will not be executed.

\subsection{Related Work}

\parahead{Conventional Representations}
%
Due to their simplicity, {\sal}s are perhaps the most typical implementation for dictionaries.
%
But because \SemInj{} is so important in simplifying proofs, {\fpf}s also see significant use---in key works such as \emph{Software Foundations}~\cite[Maps]{Pierce:SF1}---despite requiring a bit of extra overhead.

%% TODO macro?

\Cals{} seem to get little use, perhaps because working with refinement proofs might add more hassle than \SemInj{} alleviates.
%
\Cals{} are defined in the Coq standard library as \texttt{FMapList} \citep{FMapList}; a GitHub search for \texttt{FMapList} shows $304$ results,
%
whereas a search for \texttt{FMapAVL} shows $474$ results, suggesting that {\cal}s have been found to be less useful than high-performance implementations.
%
A search for \texttt{FunctionalExtensionality} \citep{FunExt}, on the other hand, turns up $4916$ results, though most of these are probably unrelated to dictionaries.%
\footnote{\hspace{0.01in}%
%
These searches and can be reproduced by URLs of the form:
\url{https://github.com/search?l=\&p=3\&q=FMapList+language\%3ACoq\&ref=advsearch\&type=Code },
replacing \texttt{FMapList} with \texttt{FMapAVL} and \texttt{FunctionalExtensionality}.
%
Accessed August 23, 2020.
%
}
%
\citet{Amorim:fmap} provides a more comprehensive treatment of {\cal}s, augmenting them with a functional interface so that the client can
%
use them as though they were {\fpf}s (note that this is different from having a \fpf{} augmented with keys).

% github searches
% funext        : 4916
% FMapInterface : 504
% FMapAVL       : 474
% FMapList      : 304

\subsection{Conclusion and Future Work}
%
This paper discusses the important properties \SemTot, \SemInj, \EqDec, and \EzDstr, and offers an implementation for dictionaries that (mostly) fulfills these properties.

Future work could consider implementations for other key utilities, such as trees or graphs, that satisfy these properties.
%
Doing so could be far more challenging.
%
It is relatively easy to simultaneously satisfy \SemTot{} and \SemInj{} for list-like structures such as dictionaries,
%
but much more awkward to do so for highly structural data such as graphs. For unlabeled graphs in particular, establishing a one-to-one correspondence between terms and
%
semantic meanings requires understanding of the graph isomorphism problem, which involves complex algebra.

% Other things to consider mentioning, but probably not
% - try to make sense of https://github.com/arthuraa/coq-utils/blob/master/theories/nominal.v
% - Coq also has FMapPositive, which is a tree based on the binary representation
% - something something explicit substitutions, Abadi et al. POPL 1990.
% - Look deeper into TDD with Idris
