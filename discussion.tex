\section{Discussion}
\label{sec:Discussion}

\nick{Part 1: Citations/discussions organized around summary table.}

\nick{Part 2: Limitations, when plain dictionaries are enough to model order-sensitive type environments.}

\nick{Part 3: Future design of more complex data structures that satisfy desirable properties like those considered for dictionaries in this paper. Blah.}

TODO TODO TODO - REFERENCES
\\
the second answer on this stack overflow suggests the finite partial function approach:
\\
https://stackoverflow.com/questions/47362451/creating-a-dictionary-map-in-coq
\\
oh hey appears that this is in software foundations too (mention of extensionality here as well)
\\
https://6826.csail.mit.edu/2019/lf/Maps.html
\\
A comprehensive, extensionality-aware hybrid of CAL and FPF is at
\\
https://github.com/arthuraa/extructures/blob/master/theories/fmap.v
\\
(also try to make sense of https://github.com/arthuraa/coq-utils/blob/master/theories/nominal.v )
\\
Coq also has CALs
\\
https://coq.inria.fr/library/Coq.FSets.FMapList.html
\\
Coq also has FMapPositive, which is a tree based on the binary representation
\\
This one seems to attach a canonicity proof to an ordinary list:
\\
http://www.cs.bc.edu/~tassarot/papers/iris-refinement/coqdoc/iris.prelude.natmap.html
\\
something something explicit substitutions, Abadi et al. POPL 1990.
\\
Look deeper into TDD with Idris

\citep{HazelnutPOPL}
