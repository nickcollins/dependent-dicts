\begin{teaserfigure}
\centering
%% \begin{figure*}
\newcommand{\lameq}[1]{$\lambda x$. $x = {#1}$}
\begin{tabular}{ l l }
 %% \Sal{} (\SAL) & [(3, "c"), (1, "b"), {\color{gray} (3, "q")}, (6, "d")] \\
 \Sal{} & [(3, "c"), (1, "b"), {\color{gray} (3, "q")}, (6, "d")] \\
 \Cal{} & [(1, "b"), (3, "c"), (6, "d")] \\ %% \quad (insert function dedupes and preserves canonical order)
 \Fpf{} & \lameq{3} ? "c" : (\lameq{1} ? "b" : (\lameq{3} ? {\color{gray} "q"} : (\lameq{6} ? "d" : $\bot$) $x$) $x$) $x$ \\
 %% \Fpfk{} & ((\lameq{3} ? "c" : (\lameq{1} ? "b" : (\lameq{3} ? {\color{gray} "q"} : (\lameq{6} ? "d" : $\bot$) $x$) $x$) $x$), [1, 3, 6])  \\
 \Dd{}  & [(1, "b"), (1, "c"), (2, "d")]
\end{tabular}
%% \caption{Dictionary representations which result from adding the sequence of keys 6, 3, 1, and 3 (again).}
%
%% \captionsetup{justification=centering}
%% \caption{Dictionary representations after mapping the sequence of keys 6, 3, 1, and 3 (again). \\
%% Each represents the same finite map: %% TODO macros if/when this stabilizes
%% \{ $1 \mapsto ``b"$, $3 \mapsto ``c"$, $6 \mapsto ``d"$ \}
%% }
%
\caption{Dictionary representations after inserting the sequence of keys 6, 3, 1, and 3:
\{ $1 \mapsto ``b"$, $3 \mapsto ``c"$, $6 \mapsto ``d"$ \}
}
%
\label{fig:intro-example}
%% \end{figure*}
\vspace{0.15in} %% SPACE HACK: between teaser figure and Abstract
%% \vspace{0.05in} %% SPACE HACK: between teaser figure and Abstract
\end{teaserfigure}
