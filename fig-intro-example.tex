\begin{teaserfigure}
\newcommand{\some}[1]
  %% {Some {#1}}
  {#1}
\newcommand{\none}[1]
  {None}
  %% {\bot}
\centering
%% \begin{figure*}
\newcommand{\lameq}[1]{$\lambda x$. $x = {#1}$}
\begin{tabular}{ l l }
 %% \Sal{} (\SAL) & [(3, "b"), (1, "a"), {\color{gray} (3, "q")}, (6, "c")] \\
 \Sal{} & [(3, \str{b}), (1, \str{a}), {\color{gray} (3, \str{q})}, (6, \str{c})] \\
 \Cal{} & [(1, \str{a}), (3, \str{b}), (6, \str{c}] \\ %% \quad (insert function dedupes and preserves canonical order)
 \Fpf{} & \lameq{3} ? \some{\str{b}} : (\lameq{1} ? \some{\str{a}} : (\lameq{3} ? {\color{gray} \some{\str{q}}} : (\lameq{6} ? \some{\str{c}} : \none{}) $x$) $x$) $x$ \\
 %% \Fpfk{} & ((\lameq{3} ? "b" : (\lameq{1} ? "a" : (\lameq{3} ? {\color{gray} "q"} : (\lameq{6} ? "c" : $\bot$) $x$) $x$) $x$), [1, 3, 6])  \\
 \Dd{}  & [(1, \str{a}), (1, \str{b}), (2, \str{c})]
\end{tabular}
%% \caption{Dictionary representations which result from adding the sequence of keys 6, 3, 1, and 3 (again).}
%
%% \captionsetup{justification=centering}
%% \caption{Dictionary representations after mapping the sequence of keys 6, 3, 1, and 3 (again). \\
%% Each represents the same finite map: %% TODO macros if/when this stabilizes
%% \{ $1 \mapsto ``b"$, $3 \mapsto ``c"$, $6 \mapsto ``d"$ \}
%% }
%
\caption{Dictionary representations after inserting the sequence of keys 6, 3, 1, and 3:
\{ $1 \mapsto$ \str{a} , $3 \mapsto$ \str{b} , $6 \mapsto$ \str{c} \}
}
%
\label{fig:intro-example}
%% \end{figure*}
\vspace{0.15in} %% SPACE HACK: between teaser figure and Abstract
%% \vspace{0.05in} %% SPACE HACK: between teaser figure and Abstract
\end{teaserfigure}
