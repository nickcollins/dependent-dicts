%% \subsection{Practical Importance}
%% \label{sec:Problem:pract}
\section{Case Study}
\label{sec:CaseStudy}

\newcommand{\altEval}[2]{E \altVdash~#1 \altRARR~#2}
\newcommand{\altEvalEnv}[3]{#1 \altVdash~#2 \altRARR~#3}
\begin{figure}[t]
%% \begin{figure*}[t]
%% \_env : Set \altRArr Set
V env = ?

data stlc : Set where
  Var\_ : string \altRArr stlc
  \altCdot\altLamb\_\altCdot\_ : string \altRArr stlc \altRArr stlc
  \_\altCirc\_ : stlc \altRArr stlc \altRArr stlc
  \_::\_ : stlc \altRArr typ \altRArr stlc
  nat_ : Nat \altRArr stlc
  assert[\_==\_] : stlc \altRArr stlc \altRArr stlc

data stlc-rslt : Set where
  [\_]\altLamb\_\altCdot\_ : stlc-rslt env \altRArr string \altRArr stlc \altRArr stlc-rslt
  nat_ : Nat \altRArr stlc-rslt
  Err : stlc-rslt

data \_\altVdash\_\altRARR\_ : stlc-rslt env \altRArr stlc \altRArr stlc-rslt \altRArr Set where
  ... standard definitions for the standard constructs ...
  EvalAsrtEq : \altFAll\{E e1 r1 e2 r2\} \altRArr
                 \altEval{e1}{r1} \altRArr
                 \altEval{e2}{r2} \altRArr
                 r1 == r2 \altRArr
                 \altEval{assert[ e1 == e2 ]}{r1}
  EvalAsrtNE : \altFAll\{E e1 r1 e2 r2\} \altRArr
                 \altEval{e1}{r1} \altRArr
                 \altEval{e2}{r2} \altRArr
                 r1 \altNE r2 \altRArr
                 \altEval{assert[ e1 == e2 ]}{Err}
  EvalApErr1 : \altFAll\{E a b\} \altRArr \altEval{a}{Err} \altRArr \altEval{a \altCirc b}{Err}
  EvalApErr2 : \altFAll\{E a b\} \altRArr \altEval{b}{Err} \altRArr \altEval{a \altCirc b}{Err}

contraction : \{E : stlc-rslt env\} \{x : string\} \{e : stlc\} \{v v' r : stlc-rslt\} \altRArr
               \altEvalEnv{(E ,, (x , v') ,, (x , v))}{e}{r} \altRArr
               \altEvalEnv{(E ,, (x , v))}{e}{r}

exchange : \{E : stlc-rslt env\} \{x1 x2 : string\} \{e : stlc\} \{v1 v2 r : stlc-rslt\} \altRArr
            x1 \altNE x2 \altRArr
            \altEvalEnv{(E ,, (x1 , v1) ,, (x2 , v2))}{e}{r} \altRArr
            \altEvalEnv{(E ,, (x2 , v2) ,, (x1 , v1))}{e}{r}

strong-norm : \altFAll\{\altGam e t\} \altRArr
                \altGam \altVdash e :: t \altRArr
                \altSum[ r \altIn stlc-rslt ] (\altEval{e}{r})


%% testbenchnarrow.tex TOO TALL WITH THIS...
%%
%% \_env : Set \altRArr Set
%% V env = ?
%%
\begin{alltt}
data exp : Set where
  Var\_         : string \altRArr exp
  \altCdot\altLamb\_\altCdot\_         : string \altRArr exp \altRArr exp
  \_\altCirc\_          : exp \altRArr exp \altRArr exp
  \_::\_         : exp \altRArr typ \altRArr exp
  nat_         : Nat \altRArr exp
  assert[\_==\_] : exp \altRArr exp \altRArr exp

data res : Set where
  [\_]\altLamb\_\altCdot\_ : res env \altRArr string \altRArr exp \altRArr res
  nat_   \hspace{0.04in}: Nat \altRArr res
  Err    \hspace{0.04in}: res

data \_\altVdash\_\altRARR\_ : res env \altRArr exp \altRArr res \altRArr Set where
  \{- Standard constructs elided... -\}

  EvalAsrtEq :
    \altFAll\{E e1 r1 e2 r2\} \altRArr
      \altEval{e1}{r1} \altRArr \altEval{e2}{r2} \altRArr r1 == r2 \altRArr
      \altEval{assert[ e1 == e2 ]}{r1}
  EvalAsrtNE :
    \altFAll\{E e1 r1 e2 r2\} \altRArr
      \altEval{e1}{r1} \altRArr \altEval{e2}{r2} \altRArr r1 \altNE r2 \altRArr
      \altEval{assert[ e1 == e2 ]}{Err}
  EvalApErr1 :
    \altFAll\{E a b\} \altRArr
      \altEval{a}{Err} \altRArr \altEval{a \altCirc b}{Err}
  EvalApErr2 :
    \altFAll\{E a b\} \altRArr
      \altEval{b}{Err} \altRArr \altEval{a \altCirc b}{Err}

contraction :
  \{E : res env\} \{x : string\}
  \{e : exp\} \{v v' r : res\} \altRArr
   \altEvalEnv{(E ,, (x , v') ,, (x , v))}{e}{r} \altRArr
   \altEvalEnv{(E ,, (x , v))}{e}{r}

exchange :
  \{E : res env\} \{x1 x2 : string\}
  \{e : exp\} \{v1 v2 r : res\} \altRArr
   x1 \altNE x2 \altRArr
   \altEvalEnv{(E ,, (x1 , v1) ,, (x2 , v2))}{e}{r} \altRArr
   \altEvalEnv{(E ,, (x2 , v2) ,, (x1 , v1))}{e}{r}

strong-norm :
  \altFAll\{\altGam e t\} \altRArr
   \altGam \altVdash e :: t \altRArr \altSum[ r \altIn res ] (\altEval{e}{r})

\end{alltt}
\caption{Mechanization requiring \dds{}.}
%% (environment, type, type context, and type-checking elided).}
%% \caption{Example mechanization scenario that requires \dds{} (type, type context, and type-checking definitions elided).}
\label{fig:testbench}
\end{figure}
%% \end{figure*}


To illustrate the practical importance of the four core properties, we consider a simply-typed $\lambda$-calculus augmented with an \texttt{assert} operator.
%
Suppose our goal is to formalize an environment-based (as opposed to substitution-based) evaluation semantics,
%
and prove that evaluation is strongly normalizing and that it satisfies the structural properties contraction and exchange.

\autoref{fig:testbench} shows a sketch of this environment-based evaluation (\texttt{\_$\vdash$\_$\Rightarrow$\_}) of expressions (\texttt{exp}) to results (\texttt{res}).
%
The definitions of evaluation environments (\texttt{env}), types, type contexts, and typechecking are not shown.
%
%; the full development is available in the supplementary materials.
%
\texttt{assert} takes two arguments; if their evaluation results are exactly equal, \texttt{assert} evaluates to the first,
otherwise it evaluates to an \texttt{Err} result.

What dictionary implementation should we choose to represent evaluation environments,
which are used both as a parameter to the evaluation judgment as well as a data component of closure results?
%
%% Below we show that \dds{} are a suitable choice, but that all the conventional solutions fail.

If we choose \sal{}s, contraction is false:
%
for example, given environments
%
$E_1=$\ \texttt{E ,, (n , v') ,, (n , v)} and
%
$E_2=$\ \texttt{E ,, (n , v)},
%
evaluation produces the closures
%
$\altEvalEnv{E_1}{\lambda x. e}{[E_1]\ \lambda x. e}$
%
and
%
$\altEvalEnv{E_2}{\lambda x. e}{[E_2]\ \lambda x. e}$
%
but
%
$[E_1]\ \lambda x. e \neq [E_2]\ \lambda x. e$.
%
A similar problem occurs for exchange, so it is also false.
%
In order for contraction and exchange to be true for any system with closure results,
the dictionary implementation used for closures must be \extensional.

\Cals{} must be packaged with validity proofs wherever they go, including in the closure result.
In such a validity proposition \texttt{valid : dict -> Set},
%
%% (defined in full in the supplementary materials)
%
the dictionary object is in negative position,
and in our case the dictionary type is dependent on the result datatype,
so the result is also in negative position, violating strict positivity.
As such, results cannot contain validity proofs, so for any system that uses dictionaries as data,
the dictionaries must be \total.

Because \fpf{}s do not have \EqDec, it is not possible to decide which \texttt{EvalAsrt} constructor would
apply to an \texttt{assert} of two \fpf{}s. As such, using \fpf{}s would make it impossible to constructively prove strong normalization.

In contrast, \dds---uniquely amongst the surveyed solutions---possess \SemTot, \SemInj, and \EqDec.
%
Thus, they are a suitable choice for implementing environments, enabling proofs of contraction, exchange, and strong normalization.
%
We discuss the generality of this case study further in \autoref{sec:Discussion:Generality}.
