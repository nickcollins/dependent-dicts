%% \subsection{Practical Importance}
%% \label{sec:Problem:pract}
\section{Case Study}
\label{sec:CaseStudy}

To illustrate the practical importance of the four core properties, we consider a simply typed lambda calculus
augmented with an \texttt{assert} operator -- see \autoref{fig:testbench}.
\texttt{assert} takes two arguments; if their evaluation results are exactly equal, \texttt{assert} evaluates to the first,
otherwise it evaluates to an \texttt{Err} result. We wish to define an environment-based (as opposed to substitution-based)
evaluation judgment, and prove that this judgment is strongly normalizing and that it  satisfies the
\emph{structural properties} \emph{contraction} and \emph{exchange}. To achieve this,
we need to choose some dictionary implementation to use for environments,
which will be used as a parameter to the evaluation judgment and as a data component in closure results.
In this section we show that \dds{} are a suitable choice, but that all the conventional solutions fail.

\newcommand{\altEval}[2]{E \altVdash~#1 \altRARR~#2}
\newcommand{\altEvalEnv}[3]{#1 \altVdash~#2 \altRARR~#3}
\begin{figure}[t]
%% \begin{figure*}[t]
%% \_env : Set \altRArr Set
V env = ?

data stlc : Set where
  Var\_ : string \altRArr stlc
  \altCdot\altLamb\_\altCdot\_ : string \altRArr stlc \altRArr stlc
  \_\altCirc\_ : stlc \altRArr stlc \altRArr stlc
  \_::\_ : stlc \altRArr typ \altRArr stlc
  nat_ : Nat \altRArr stlc
  assert[\_==\_] : stlc \altRArr stlc \altRArr stlc

data stlc-rslt : Set where
  [\_]\altLamb\_\altCdot\_ : stlc-rslt env \altRArr string \altRArr stlc \altRArr stlc-rslt
  nat_ : Nat \altRArr stlc-rslt
  Err : stlc-rslt

data \_\altVdash\_\altRARR\_ : stlc-rslt env \altRArr stlc \altRArr stlc-rslt \altRArr Set where
  ... standard definitions for the standard constructs ...
  EvalAsrtEq : \altFAll\{E e1 r1 e2 r2\} \altRArr
                 \altEval{e1}{r1} \altRArr
                 \altEval{e2}{r2} \altRArr
                 r1 == r2 \altRArr
                 \altEval{assert[ e1 == e2 ]}{r1}
  EvalAsrtNE : \altFAll\{E e1 r1 e2 r2\} \altRArr
                 \altEval{e1}{r1} \altRArr
                 \altEval{e2}{r2} \altRArr
                 r1 \altNE r2 \altRArr
                 \altEval{assert[ e1 == e2 ]}{Err}
  EvalApErr1 : \altFAll\{E a b\} \altRArr \altEval{a}{Err} \altRArr \altEval{a \altCirc b}{Err}
  EvalApErr2 : \altFAll\{E a b\} \altRArr \altEval{b}{Err} \altRArr \altEval{a \altCirc b}{Err}

contraction : \{E : stlc-rslt env\} \{x : string\} \{e : stlc\} \{v v' r : stlc-rslt\} \altRArr
               \altEvalEnv{(E ,, (x , v') ,, (x , v))}{e}{r} \altRArr
               \altEvalEnv{(E ,, (x , v))}{e}{r}

exchange : \{E : stlc-rslt env\} \{x1 x2 : string\} \{e : stlc\} \{v1 v2 r : stlc-rslt\} \altRArr
            x1 \altNE x2 \altRArr
            \altEvalEnv{(E ,, (x1 , v1) ,, (x2 , v2))}{e}{r} \altRArr
            \altEvalEnv{(E ,, (x2 , v2) ,, (x1 , v1))}{e}{r}

strong-norm : \altFAll\{\altGam e t\} \altRArr
                \altGam \altVdash e :: t \altRArr
                \altSum[ r \altIn stlc-rslt ] (\altEval{e}{r})


%% testbenchnarrow.tex TOO TALL WITH THIS...
%%
%% \_env : Set \altRArr Set
%% V env = ?
%%
\begin{alltt}
data exp : Set where
  Var\_         : string \altRArr exp
  \altCdot\altLamb\_\altCdot\_         : string \altRArr exp \altRArr exp
  \_\altCirc\_          : exp \altRArr exp \altRArr exp
  \_::\_         : exp \altRArr typ \altRArr exp
  nat_         : Nat \altRArr exp
  assert[\_==\_] : exp \altRArr exp \altRArr exp

data res : Set where
  [\_]\altLamb\_\altCdot\_ : res env \altRArr string \altRArr exp \altRArr res
  nat_   \hspace{0.04in}: Nat \altRArr res
  Err    \hspace{0.04in}: res

data \_\altVdash\_\altRARR\_ : res env \altRArr exp \altRArr res \altRArr Set where
  \{- Standard constructs elided... -\}

  EvalAsrtEq :
    \altFAll\{E e1 r1 e2 r2\} \altRArr
      \altEval{e1}{r1} \altRArr \altEval{e2}{r2} \altRArr r1 == r2 \altRArr
      \altEval{assert[ e1 == e2 ]}{r1}
  EvalAsrtNE :
    \altFAll\{E e1 r1 e2 r2\} \altRArr
      \altEval{e1}{r1} \altRArr \altEval{e2}{r2} \altRArr r1 \altNE r2 \altRArr
      \altEval{assert[ e1 == e2 ]}{Err}
  EvalApErr1 :
    \altFAll\{E a b\} \altRArr
      \altEval{a}{Err} \altRArr \altEval{a \altCirc b}{Err}
  EvalApErr2 :
    \altFAll\{E a b\} \altRArr
      \altEval{b}{Err} \altRArr \altEval{a \altCirc b}{Err}

contraction :
  \{E : res env\} \{x : string\}
  \{e : exp\} \{v v' r : res\} \altRArr
   \altEvalEnv{(E ,, (x , v') ,, (x , v))}{e}{r} \altRArr
   \altEvalEnv{(E ,, (x , v))}{e}{r}

exchange :
  \{E : res env\} \{x1 x2 : string\}
  \{e : exp\} \{v1 v2 r : res\} \altRArr
   x1 \altNE x2 \altRArr
   \altEvalEnv{(E ,, (x1 , v1) ,, (x2 , v2))}{e}{r} \altRArr
   \altEvalEnv{(E ,, (x2 , v2) ,, (x1 , v1))}{e}{r}

strong-norm :
  \altFAll\{\altGam e t\} \altRArr
   \altGam \altVdash e :: t \altRArr \altSum[ r \altIn res ] (\altEval{e}{r})

\end{alltt}
\caption{Mechanization requiring \dds{}.}
%% (environment, type, type context, and type-checking elided).}
%% \caption{Example mechanization scenario that requires \dds{} (type, type context, and type-checking definitions elided).}
\label{fig:testbench}
\end{figure}
%% \end{figure*}


For \sal{}s, \emph{contraction} is false --
\altEvalEnv{\hbox{E ,, (n , v') ,, (n , v)}}{\hbox{$\cdot\lambda x \cdot e$}}{\hbox{[E ,, (n , v') ,, (n , v)]$\lambda x \cdot e$}}
while \altEvalEnv{E ,, (n , v)}{\hbox{$\cdot\lambda x \cdot e$}}{\hbox{[E ,, (n , v)]$\lambda x \cdot e$}}, yet
\hbox{[E ,, (n , v') ,, (n , v)]$\lambda x \cdot e$}$~\ne~$\hbox{[E ,, (n , v)]$\lambda x \cdot e$}.
A similar problem occurs for \emph{exchange}, so it is also false.
In order for \emph{contraction} and \emph{exchange} to be true for any system with closure results,
the dictionary implementation used for closures must be \extensional.

\Cals{} must be packaged with validity proofs wherever they go, including in the closure result.
In the validity proposition \texttt{valid : dict -> Set}, the dictionary object is in negative position,
and in our case the dictionary type is dependent on the result datatype,
so \texttt{utlc-result} is also in negative position, violating strict positivity.
As such, results cannot contain validity proofs, so for any system that uses dictionaries as data,
the dictionaries must be \total.

Because \fpf{}s do not have \EqDec, it's not possible to decide which \texttt{EvalAsrt} constructor would
apply to an \texttt{assert} of two \fpf{}s. As such, it's not possible to constructively prove strong normalization.

Because \dds, uniquely amongst the surveyed solutions, possess \SemTot, \SemInj, and \EqDec,
they are a suitable choice for implementing environments, enabling proofs of \emph{contraction},
\emph{exchange}, and strong normalization.

This scenario may seem contrived, but it is actually a simplification of a real task faced by the authors.
The task was to mechanize the formalization of Smyth\citep{smyth}, a program synthesis technique which uses
natural semantics (\ie{} big-step environment-based evaluation) and which requires assertions.
Although this mechanization has not been completed, challenges (related to the aforementioned issues)
faced during its development necessitated the invention of \dds.

Perhaps, instead of using \dds, these issues could be worked around, by changing or dropping some of our criteria?
In many cases, substitution can be used to avoid environments -- however, environments are often preferred because
they allow the formalization to more closely resemble the implementation \nick{TODO source}. Furthermore,
Smyth has hole closures in addition to lambda closures, so closing over environments would be necessary regardless.

\emph{Contraction} and \emph{exchange} are not always necessary properties for program foundation judgments --
rather, substructural logics, by definition, deliberately violate one or both of these properties%
\footnote{Technically, they could retain these properties but instead violate \emph{weakening}.}.
That said, languages should generally uphold the \emph{structural properties} unless there is a strong and explicit
reason not to. In most cases, the inability to prove the \emph{structural properties} would raise a large red flag,
suggesting that there may be a bug in the language definition.

It may not seem useful to assert that two functions are intensionally identical, but if assertion is relaxed,
so as to test consistency or partially extensional equality instead of purely intensional equality,
it would require environments that are destructible.
Scenarios which require destructibility but not \EqDec{} lead to the same conclusions,
seeing as these columns are identical in \autoref{fig:prop-summary}.
