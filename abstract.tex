% !TEX root = ?.tex

%% no citations in Abstract
%%
%% \cite{?}

\begin{abstract}
Dictionaries (a.k.a. finite maps) are a core tool in any programming environment, serving many purposes.
In proof assistants - where simplicity is important but performance isn't - a typical implementation
involves a list of key-value pairs. Although easy to define and reason about, this approach has the disadvantage that keys
can be duplicated, leading to the fact that many distinct lists represent the same semantic mapping.
This many-one relationship creates various difficulties when used in practice, such as when proving
contraction and exchange for a type-checking judgment.
We introduce two (TODO three???) novel (TODO are they???) alternatives that are agnostic to insertion order and which
deduplicate bindings. Each approach has its pros and cons - chiefly, the new approaches yield contraction and exchange
for free, while destruction and iteration is easiest with the conventional approach. We also discuss the meta-theory of
dictionaries, which we've defined and proved in the Agda proof assistant.
\end{abstract}
