% !TEX root = ?.tex

%% no citations in Abstract
%%
%% \cite{?}

\begin{abstract}
Dictionaries (a.k.a. finite maps) are a core tool in any programming environment.
Conventional languages implement dictionaries with hashtables or auto-balanced trees, but proof assistants
- which favor simplicity and provability over performance - generally use association lists
(with or without canonical ordering) or finite partial functions. These solutions are suitable for many
cases - however, each one has a drawback: unordered lists can contain duplicates or have arbitrary ordering,
canonically ordered lists may be invalid and so must be refined with a proof of validity, and equality of finite partial
functions is undecidable. Using a variant of delta-encoding, we develop a novel list-based solution that preserves
canonical ordering while ensuring that every type-correct list is semantically valid, eliminating the need for refinement
with a proof of validity. Our solution also establishes a one-to-one correspondence between dictionary terms and semantic
mappings. We demonstrate the practical importance of these properties when developing metatheory for environment-based
evaluation semantics, while acknowledging that drawbacks of our solution may make it inferior to one of the conventional
solutions in some cases where that conventional solution is viable. We prove the relevant metatheory in Agda.
%
% Our solution is not suitable for all key types, and is harder to destruct or
%iterate than other list-of-pair solutions, but is nonetheless better suited to many large, complex program
%foundations metatheories than the existing solutions.
%
% Although easy to define and reason about, this approach has the disadvantage that keys
%can be duplicated, leading to the fact that many distinct lists represent the same semantic mapping.
%This many-one relationship creates various difficulties when used in practice, such as when proving
%contraction and exchange for a type-checking judgment.
\end{abstract}
