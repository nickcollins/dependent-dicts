\PassOptionsToPackage{svgnames,dvipsnames,svgnames}{xcolor}

\documentclass[nonacm]{acmart}
\settopmatter{printfolios=true,printccs=false,printacmref=false}

%% Conference information
%% Supplied to authors by publisher for camera-ready submission;
%% use defaults for review submission.
%% ?

%% Copyright information
%% Supplied to authors (based on authors' rights management selection;
%% see authors.acm.org) by publisher for camera-ready submission;
%% use 'none' for review submission.
\setcopyright{none}
%\setcopyright{acmcopyright}
%\setcopyright{acmlicensed}
%\setcopyright{rightsretained}
%\copyrightyear{2019}           %% If different from \acmYear

%% Bibliography style
\bibliographystyle{ACM-Reference-Format}
%% Citation style
%\citestyle{acmauthoryear}  %% For author/year citations
%\citestyle{acmnumeric}     %% For numeric citations
%\setcitestyle{nosort}      %% With 'acmnumeric', to disable automatic
                            %% sorting of references within a single citation;
                            %% e.g., \cite{Smith99,Carpenter05,Baker12}
                            %% rendered as [14,5,2] rather than [2,5,14].
%\setcitesyle{nocompress}   %% With 'acmnumeric', to disable automatic
                            %% compression of sequential references within a
                            %% single citation;
                            %% e.g., \cite{Baker12,Baker14,Baker16}
                            %% rendered as [2,3,4] rather than [2-4].


%% Some recommended packages.
\usepackage{booktabs}   %% For formal tables:
                        %% http://ctan.org/pkg/booktabs
\usepackage{subcaption} %% For complex figures with subfigures/subcaptions
                        %% http://ctan.org/pkg/subcaption

%% Cyrus packages
\usepackage{microtype}
\usepackage{mdframed}
\usepackage{colortab}
\usepackage{mathpartir}
\usepackage{enumitem}
\usepackage{bbm}
\usepackage{stmaryrd}
\usepackage{mathtools}
\usepackage{leftidx}
\usepackage{todonotes}
\usepackage{xspace}
\usepackage{wrapfig}

\usepackage{listings}%
\lstloadlanguages{ML}
\lstset{tabsize=2, 
basicstyle=\footnotesize\ttfamily, 
% keywordstyle=\sffamily,
commentstyle=\itshape\ttfamily\color{gray}, 
stringstyle=\ttfamily\color{purple},
mathescape=false,escapechar=\#,
numbers=left, numberstyle=\scriptsize\color{gray}\ttfamily, language=ML, showspaces=false,showstringspaces=false,xleftmargin=10pt, 
morekeywords={string, float, int},
classoffset=0,belowskip=\smallskipamount, aboveskip=\smallskipamount,
moredelim=**[is][\color{red}]{SSTR}{ESTR}
}
\newcommand{\li}[1]{\lstinline[basicstyle=\ttfamily\fontsize{9pt}{1em}\selectfont]{#1}}
\newcommand{\lismall}[1]{\lstinline[basicstyle=\ttfamily\fontsize{9pt}{1em}\selectfont]{#1}}

%% Joshua Dunfield macros
\def\OPTIONConf{1}%
\usepackage{joshuadunfield}

%% Can remove this eventually
\usepackage{blindtext}

\usepackage{enumitem}

% \newtheorem{theorem}{Theorem}[chapter]
% \newtheorem{lemma}[theorem]{Lemma}
% \newtheorem{corollary}[theorem]{Corollary}
% \newtheorem{definition}[theorem]{Definition}
% \newtheorem{assumption}[theorem]{Assumption}
% \newtheorem{condition}[theorem]{Condition}

\newtheoremstyle{slplain}% name
  {.15\baselineskip\@plus.1\baselineskip\@minus.1\baselineskip}% Space above
  {.15\baselineskip\@plus.1\baselineskip\@minus.1\baselineskip}% Space below
  {\slshape}% Body font
  {\parindent}%Indent amount (empty = no indent, \parindent = para indent)
  {\bfseries}%  Thm head font
  {.}%       Punctuation after thm head
  { }%      Space after thm head: " " = normal interword space;
        %       \newline = linebreak
  {}%       Thm head spec
\theoremstyle{slplain}
\newtheorem{thm}{Theorem}  % Numbered with the equation counter
\numberwithin{thm}{section}
\newtheorem{defn}[thm]{Definition}
\newtheorem{lem}[thm]{Lemma}
\newtheorem{prop}[thm]{Proposition}
% \newtheorem{cor}[section]{Corollary}     
% \newtheorem{lem}[section]{Lemma}         
% \newtheorem{prop}[section]{Proposition}  

% \setlength{\abovedisplayskip}{0pt}
% \setlength{\belowdisplayskip}{0pt}
% \setlength{\abovedisplayshortskip}{0pt}
% \setlength{\belowdisplayshortskip}{0pt}

\fancyfoot{} % suppresses the footer (also need \thispagestyle{empty} after \maketitle below)

% !TEX root = main.tex

\newcommand{\mynote}[3]{\textcolor{#3}{\textsf{{#2}}}}
\newcommand{\rkc}[1]{\mynote{rkc}{#1}{blue}}
\newcommand{\nick}[1]{\mynote{cy}{#1}{purple}}

\newcommand{\cvert}{{\,{\vert}\,}}

%% https://tex.stackexchange.com/questions/9796/how-to-add-todo-notes
\newcommand{\rkcTodo}[1]{\todo[linecolor=blue,backgroundcolor=blue!25,bordercolor=blue]{#1}}

\newcommand{\mattTodo}[1]{\todo[linecolor=green,backgroundcolor=green!2,bordercolor=green]{\tiny\textit{#1}}}
\newcommand{\mattOmit}[1]{\colorbox{yellow}{(Matt omitted stuff here)}}

\def\parahead#1{\paragraph{\textbf{#1.}}}
%% \def\paraheadNoDot#1{\paragraph{{\textbf{#1}}}}
\def\subparahead#1{\paragraph{\textit{#1.}}}
%% \def\paraheadindent#1{\paragraph{}\textit{#1.}}
%% \def\paraheadindentnodot#1{\paragraph{}\textit{#1}}

\newcommand{\ie}{i.e.}
\newcommand{\eg}{e.g.}
% \newcommand{\etc}{{\emph{etc.}}}
% \newcommand{\cf}{{\emph{cf.}}}
% \newcommand{\etal}{{\emph{et al.}}}

%% \newcommand{\hazel}{\ensuremath{\textsc{Hazel}}}
%% \newcommand{\sns}{\ensuremath{\textsc{Sketch-n-Sketch}}}
%% \newcommand{\deuce}{\ensuremath{\textsc{Deuce}}}
\newcommand{\Elm}{\ensuremath{\textsf{Elm}}}
\newcommand{\sns}{\ensuremath{\textrm{Sketch-n-Sketch}}}
\newcommand{\deuce}{\ensuremath{\textrm{Deuce}}}

\newcommand{\sectionDescription}[1]{\section{#1}}
\newcommand{\subsectionDescription}[1]{\subsection{#1}}
\newcommand{\subsubsectionDescription}[1]{\subsubsection{#1}}
%% \newcommand{\subsectionDescription}[1]{\subsection*{#1}}
\newcommand{\suppMaterials}{the Supplementary Materials}

\newcommand{\defeq}{\overset{\textrm{def}}{=}}

\newcommand{\eap}{action suggestion panel\xspace}
\newcommand{\Eap}{Action suggestion panel\xspace}

\newcommand{\myfootnote}[1]{\footnote{ #1}}

\def\sectionautorefname{Section}
\def\subsectionautorefname{Section}
\def\subsubsectionautorefname{Section}

\newcommand{\code}[1]{\lstinline{#1}}
\newcommand{\str}[1]
  {``#1''}

% Make italic?
%\newcommand{\Property}[1]{\emph{#1}}
\newcommand{\Property}[1]{\textrm{#1}}

% Calling out Cyrus's favorite verb, 'to be' ;)
\newcommand{\IS}{\colorbox{red}{is}\xspace}

\newcommand{\codeSize}
  %% {\footnotesize}
  {\small}

%\newcommand{\JoinTypes}[2]{\textsf{join}~~#1~~#2}
\newcommand{\JoinTypes}[2]{\textsf{join}(#1,#2)}

%% \newtheorem{theorem}{Theorem}[section] % 2.1, 2.2, etc.
\newtheorem{theorem}{Theorem}             % 1, 2, etc.
\newtheorem{remark}[theorem]{Remark}

%%%%%%%%%%%%%%%%%%%%%%%%%%%%%%%%%%%%%%%%%%%%%%%%%%%%%%%%%%%%%%%%%%%%%%%%%%%%%%%%
%% Spacing

\newcommand{\sep}{\hspace{0.06in}}
\newcommand{\sepPremise}{\hspace{0.20in}}
\newcommand{\hsepRule}{\hspace{0.20in}}
\newcommand{\vsepRuleHeight}{0.08in}
\newcommand{\vsepRule}{\vspace{\vsepRuleHeight}}
\newcommand{\miniSepOne}{\hspace{0.01in}}
\newcommand{\miniSepTwo}{\hspace{0.02in}}
\newcommand{\miniSepThree}{\hspace{0.03in}}
\newcommand{\miniSepFour}{\hspace{0.04in}}
\newcommand{\miniSepFive}{\hspace{0.05in}}
\newcommand{\breakAndIndent}
  {\mbox{}

   %% \hspace{0.15in}
   \hspace{0.00in}
  }

\newcommand{\justIndent}
  %% {\hspace{0.15in}
  {\hspace{0.00in}
  }

%%%%%%%%%%%%%%%%%%%%%%%%%%%%%%%%%%%%%%%%%%%%%%%%%%%%%%%%%%%%%%%%%%%%%%%%%%%%%%%%

% \lstset{
% %mathescape=true,basicstyle=\fontsize{8}{9}\ttfamily,
% literate={=>}{$\Rightarrow$}2
%          {<=}{$\leq$}2
%          {->}{${\rightarrow}$}1
%          {\\\\=}{\color{red}{$\lambda$}}2
%          {\\\\}{$\lambda$}2
%          {**}{$\times$}2
%          {*.}{${\color{blue}{\texttt{*.}}}$}2
%          {+.}{${\color{blue}{\texttt{+.}}}$}2
%          {<}{${\color{green}{\lhd}}$}1
%          {>?}{${\color{green}{\rhd}}$?}2
%          {<<}{${\color{green}{\blacktriangleleft}}$}1
%          {>>?}{${\color{green}{\blacktriangleright}}$?}2
%          {\{}{${\color{blue}{\{}}$}1
%          {\}}{${\color{blue}{\}}}$}1
%          {[}{${\color{purple}{[}}$}1
%          {]}{${\color{purple}{]}}$}1
%          {(}{${\color{darkgray}{\texttt{(}}}$}1
%          {)}{${\color{darkgray}{\texttt{)}}}$}1
%          {]]}{${\color{gray}{\big(}}$}1
%          {]]}{${\color{gray}{\big)}}$}1
% }

%%%%%%%%%%%%%%%%%%%%%%%%%%%%%%%%%%%%%%%%%%%%%%%%%%%%%%%%%%%%%%%%%%%%%%%%%%%%%%%%

\newcommand{\li}[1]{\lstinline[basicstyle=\ttfamily\fontsize{9pt}{1em}\selectfont]{#1}}
\newcommand{\lismall}[1]{\lstinline[basicstyle=\ttfamily\fontsize{9pt}{1em}\selectfont]{#1}}
\newcommand{\mapP}{\{0 \rightarrow \text{"a"}, 1 \rightarrow \text{"b"}, 3 \rightarrow \text{"c"}, 6 \rightarrow \text{"d"}, 10 \rightarrow \text{"e"}\}}
\newcommand{\dictP}{[(0, "a"), (0, "b"), (1, "c"), (2, "d"), (3, "e")]}

%%%%%%%%%%%%%%%%%%%%%%%%%%%%%%%%%%%%%%%%%%%%%%%%%%%%%%%%%%%%%%%%%%%%%%%%%%%%%%%%

%% \newcommand{\sal}{simple association list}
\newcommand{\sal}{association list}
\newcommand{\Sal}{Association List}
\newcommand{\SAL}{AL}

%% \newcommand{\cal}{canonicalized association list}
\newcommand{\cal}{canonical association list}
\newcommand{\Cal}{Canonical Association List}
\newcommand{\CAL}{CAL}
\newcommand{\Cals}{Canonical association lists}

%% \newcommand{\fpf}{finite partial function}
\newcommand{\fpf}{partial function}
\newcommand{\Fpf}{Partial Function}
\newcommand{\Fpfs}{Partial functions}
\newcommand{\FPF}{PF}

%% trying this out...
%% \newcommand{\cfpf}{canon. finite partial function}
\newcommand{\fpfk}{partial function with keys}
\newcommand{\Fpfk}{Partial Function with Keys}
\newcommand{\FPFK}{PFK}

\newcommand{\dd}{delta dictionary}
\newcommand{\dds}{delta dictionaries}
\newcommand{\Dd}{Delta Dictionary}
\newcommand{\Ddls}{Delta dictionaries}
\newcommand{\DD}{DD}

%%%%%%%%%%%%%%%%%%%%%%%%%%%%%%%%%%%%%%%%%%%%%%%%%%%%%%%%%%%%%%%%%%%%%%%%%%%%%%%%

\newcommand{\SemTot}{Semantic Totality}
%% \newcommand{\SemInj}{Semantic Injectivity}
\newcommand{\SemInj}{Extensionality}
\newcommand{\EqDec}{Decidable Equality}
%% \newcommand{\EzDstr}{Ease of Destruction}
\newcommand{\EzDstr}{Easy Destructibility}

%% tweak and remove indirection macros later, once terminology settles
\newcommand{\Total}{\SemTot}
\newcommand{\Extensional}{\SemInj}
\newcommand{\DecidableEq}{\EqDec}
\newcommand{\Destructible}{\EzDstr}

\newcommand{\total}{Total}
\newcommand{\extensional}{Extensional}
\newcommand{\decidable}{Decid. Eq.}
\newcommand{\destructible}{Destructible}

%%%%%%%%%%%%%%%%%%%%%%%%%%%%%%%%%%%%%%%%%%%%%%%%%%%%%%%%%%%%%%%%%%%%%%%%%%%%%%%%

% for use in alltt
\newcommand{\altFAll}{\ensuremath{\forall}}
\newcommand{\altRArr}{\ensuremath{\rightarrow}}
\newcommand{\altRARR}{\ensuremath{\Rightarrow}}
\newcommand{\altVdash}{\ensuremath{\vdash}}
\newcommand{\altAnd}{\ensuremath{\land}}
\newcommand{\altOr}{\ensuremath{\lor}}
\newcommand{\altNE}{\ensuremath{\ne}}
\newcommand{\altEmpty}{\ensuremath{\varnothing}}
\newcommand{\altIn}{\ensuremath{\in}}
\newcommand{\altNIn}{\ensuremath{\notin}}
\newcommand{\altSum}{\ensuremath{\sum}}
\newcommand{\altLAng}{\ensuremath{\langle}}
\newcommand{\altRAng}{\ensuremath{\rangle}}
\newcommand{\altLamb}{\ensuremath{\lambda}}
\newcommand{\altGam}{\ensuremath{\Gamma}}
\newcommand{\altCirc}{\ensuremath{\circ}}
\newcommand{\altCdot}{\ensuremath{\cdot}}

% !TEX root = hazelnut-dynamics.tex

% \newcommand{\Label}[1]{\vspace{-20px}\label{#1}%
%   {\small\textcolor{cyan}{(\texttt{#1})}}\vspace{20px}%
% }

\newcommand{\cmttclo}[2]{\mathsf{clo}(#1, #2)}

% \newcommand{\CaptionLabel}[2]{
%   \caption{#1 {\small\textcolor{cyan}{(#2)}}}
%   \label{#2}}
\newcommand{\CaptionLabel}[2]{
  \caption{#1}
  \label{#2}}

% Violet hotdogs; highlight color helps distinguish them
\newcommand{\llparenthesiscolor}{\textcolor{violet}{\llparenthesis}}
\newcommand{\rrparenthesiscolor}{\textcolor{violet}{\rrparenthesis}}
% \newcommand{\llparenthesiscolor}{\textcolor{red}{\lfloor}}
% \newcommand{\rrparenthesiscolor}{\textcolor{red}{\rfloor}}

%% TODO if feeling really obsessive, use the following in place of x,u,c,b
\newcommand{\varVar}{x}
\newcommand{\varHole}{u}
\newcommand{\econst}{c}
\newcommand{\tbase}{b}

% HTyp and HExp
\newcommand{\isComplete}[1]{#1~\mathsf{complete}}

% HTyp
\newcommand{\htau}{\tau}
\newcommand{\tarr}[2]{#1 \rightarrow #2}
%\newcommand{\tsum}[2]{#1 + #2}
\newcommand{\tprod}[2]{#1 \times #2}
\newcommand{\tnum}{\texttt{num}}
\newcommand{\tb}{\texttt{b}}
\newcommand{\tehole}{\llparenthesiscolor\rrparenthesiscolor}
\newcommand{\tsum}[2]{{#1} + {#2}}

\newcommand{\tconsistent}[2]{#1 \sim #2}
\newcommand{\tinconsistent}[2]{#1 \nsim #2}

% HExp
\newcommand{\hexp}{e}
\newcommand{\hlam}[2]{\lambda #1.#2}
\newcommand{\halam}[3]{\lambda #1{:}#2.#3}
\newcommand{\hap}[2]{#1(#2)}
\newcommand{\hapP}[2]{(#1)~(#2)} % Extra paren around function term
\newcommand{\hpair}[2]{(#1, #2)}
\newcommand{\hprj}[2]{\mathsf{prj}_{#1}(#2)}
\newcommand{\lblL}{\mathsf{L}}
\newcommand{\lblR}{\mathsf{R}}
\newcommand{\hnum}[1]{\underline{#1}}
%\newcommand{\hcase}[5]{\mathsf{case}\,#1\,\mathsf{of}\,#2\Rightarrow#3~\vert~#4\Rightarrow#5}
\newcommand{\hadd}[2]{#1 + #2}
\newcommand{\hehole}[1]{\llparenthesiscolor\rrparenthesiscolor^{#1}}
% \newcommand{\hhole}[1]{\setlength{\fboxsep}{0pt}\fcolorbox{red}{white}{\vphantom{)}$#1$}}
\newcommand{\hhole}[2]{\llparenthesiscolor#1\rrparenthesiscolor^{#2}}
% \newcommand{\hhole}[1]{
  % \setlength{\fboxsep}{0pt}
  % \colorbox{violet!10!white!100}{\ensuremath{\llparenthesiscolor#1\rrparenthesiscolor}}}
\newcommand{\hindet}[1]{\lceil#1\rceil}
%\newcommand{\hinj}[2]{\texttt{inj}_{#1}({#2})}
\newcommand{\hinL}[1]{\mathsf{inl}(#1)}
\newcommand{\hinR}[1]{\mathsf{inr}(#1)}
\newcommand{\hcase}[5]{\texttt{case}({#1},{#2}.{#3},{#4}.{#5})}

\newcommand{\hGamma}{\Gamma}
\newcommand{\EmptyhGamma}{\emptyset} % From hand-written notes, Canonical forms lemma; page 14
\newcommand{\EmptyDelta}{\cdot} % From hand-written notes, ES-Const rule, page 1
\newcommand{\domof}[1]{\text{dom}(#1)}
\newcommand{\hsyn}[3]{#1 \vdash #2 \Rightarrow #3}
\newcommand{\hana}[3]{#1 \vdash #2 \Leftarrow #3}

% ZTyp and ZExp
\newcommand{\zlsel}[1]{{\bowtie}{#1}}
\newcommand{\zrsel}[1]{{#1}{\bowtie}}

%\newcommand{\zwsel}[1]{\adjustbox{cframe=blue}{\ensuremath{{\textcolor{blue}{\triangleright}}{#1}{\textcolor{blue}{\triangleleft}}}}}
\newcommand{\zwsel}[1]{
  \setlength{\fboxsep}{0pt}
  \colorbox{green!10!white!100}{
    \ensuremath{{{\textcolor{Green}{{\hspace{-2px}\triangleright}}}}{#1}{\textcolor{Green}{\triangleleft{\vphantom{\tehole}}}}}}
}
%\newcommand{\zwsel}[1]{{\triangleright}{#1}{\triangleleft}}

\newcommand{\removeSel}[1]{#1^{\diamond}}

% ZTyp
\newcommand{\ztau}{\hat{\tau}}

% ZExp
\newcommand{\zexp}{\hat{e}}

% Direction
\newcommand{\dParent}{\mathtt{parent}}
\newcommand{\dChild}{\mathtt{firstChild}}
\newcommand{\dNext}{\mathtt{nextSib}}
\newcommand{\dPrev}{\mathtt{prevSib}}

% Action
\newcommand{\aMove}[1]{\mathtt{move}~#1}
	\newcommand{\zrightmost}[1]{\mathsf{rightmost}(#1)}
	\newcommand{\zleftmost}[1]{\mathsf{leftmost}(#1)}
\newcommand{\aSelect}[1]{\mathtt{sel}~#1}
\newcommand{\aDel}{\mathtt{del}}
\newcommand{\aReplace}[1]{\mathtt{replace}~#1}
\newcommand{\aConstruct}[1]{\mathtt{construct}~#1}
\newcommand{\aConstructx}[1]{#1}
\newcommand{\aFinish}{\mathtt{finish}}

\newcommand{\performAna}[5]{#1 \vdash #2 \xlongrightarrow{#4} #5 \Leftarrow #3}
\newcommand{\performAnaI}[5]{#1 \vdash #2 \xlongrightarrow{#4}\hspace{-3px}{}^{*}~ #5 \Leftarrow #3}
\newcommand{\performSyn}[6]{#1 \vdash #2 \Rightarrow #3 \xlongrightarrow{#4} #5 \Rightarrow #6}
\newcommand{\performSynI}[6]{#1 \vdash #2 \Rightarrow #3 \xlongrightarrow{#4}\hspace{-3px}{}^{*}~ #5 \Rightarrow #6}
\newcommand{\performTyp}[3]{#1 \xlongrightarrow{#2} #3}
\newcommand{\performTypI}[3]{#1 \xlongrightarrow{#2}\hspace{-3px}{}^{*}~#3}

\newcommand{\performMove}[3]{#1 \xlongrightarrow{#2} #3}
\newcommand{\performDel}[2]{#1 \xlongrightarrow{\aDel} #2}

% Form
\newcommand{\farr}{\mathtt{arrow}}
\newcommand{\fnum}{\mathtt{num}}
\newcommand{\fsum}{\mathtt{sum}}

\newcommand{\fasc}{\mathtt{asc}}
\newcommand{\fvar}[1]{\mathtt{var}~#1}
\newcommand{\flam}[1]{\mathtt{lam}~#1}
\newcommand{\fap}{\mathtt{ap}}
\newcommand{\farg}{\mathtt{arg}}
\newcommand{\fnumlit}[1]{\mathtt{lit}~#1}
\newcommand{\fplus}{\mathtt{plus}}
\newcommand{\fhole}{\mathtt{hole}}
\newcommand{\fnehole}{\mathtt{nehole}}

\newcommand{\finj}[1]{\mathtt{inj}~#1}
\newcommand{\fcase}[2]{\mathtt{case}~#1~#2}

% Talk about formal rules in example
\newcommand{\refrule}[1]{\textrm{Rule~(#1)}}

\newcommand{\herase}[1]{\left|#1\right|_\textsf{erase}}

\newcommand{\arrmatch}[2]{#1 \blacktriangleright_{\rightarrow} #2}
%% TODO maybe write underbracket
%% \newcommand{\groundmatch}[2]{\underline{#1} = #2}
\newcommand{\groundmatch}[2]{#1 \blacktriangleright_{\mathsf{ground}} #2}
\newcommand{\prodmatch}[2]{#1 \blacktriangleright_{\times} #2}
\newcommand{\summatch}[2]{#1 \blacktriangleright_{+} #2}


\newcommand{\TABperformAna}[5]{#1 \vdash & #2                & \xlongrightarrow{#4} & #5 & \Leftarrow #3}
\newcommand{\TABperformSyn}[6]{#1 \vdash & #2 \Rightarrow #3 & \xlongrightarrow{#4} & #5 \Rightarrow #6}
\newcommand{\TABperformTyp}[3]{& #1 & \xlongrightarrow{#2} & #3}

\newcommand{\TABperformMove}[3]{#1 & \xlongrightarrow{#2} & #3}
\newcommand{\TABperformDel}[2]{#1 \xlongrightarrow{\aDel} #2}

\newcommand{\sumhasmatched}[2]{#1 \mathrel{\textcolor{black}{\blacktriangleright_{+}}} #2}

%%%% DYNAMICS %%%%
%% TODO remove these macros
%% marks for eval
\newcommand{\unevaled}{\times}
\newcommand{\evaled}{\checkmark}
\newcommand{\markname}{m}

\newcommand{\mvar}[0]{u}
\newcommand{\subst}[0]{\sigma}
\newcommand{\substitute}[3]{[#1/#2]#3}
\newcommand{\fvof}[1]{\mathsf{FV}(#1)}
\newcommand{\dexp}[0]{d}
\newcommand{\dconst}[0]{c}
\newcommand{\dval}[0]{\ddot{v}}
%% TODO remove this macro
\newcommand{\dcast}[2]{\langle #1 \rangle ~ #2}
%% TODO make the following two look better
\newcommand{\dcasttwo}[3]{#1 \langle{#2}\Rightarrow{#3}\rangle}
\newcommand{\dcastthree}[4]
  {#1 \langle{#2}\Rightarrow{#3}\Rightarrow{#4}\rangle} %% sugared version
  %% {\dcasttwo{\dcasttwo{#1}{#2}{#3}}{#3}{#4}} %% unsugared version
\newcommand{\dcastfail}[3]{#1 \langle{#2}\Rightarrow{\tehole}\not\Rightarrow{#3}\rangle}
%% \newcommand{\dlam}[3]{\lambda #1:#2.#3}
\newcommand{\dlam}[3]{\halam{#1}{#2}{#3}}
\newcommand{\dap}[2]{#1(#2)}
\newcommand{\dapP}[2]{(#1)(#2)} % Extra paren around function term
\newcommand{\dnum}[1]{\underline{#1}}
%\newcommand{\dcase}[5]{\mathsf{case}\,#1\,\mathsf{of}\,#2\Rightarrow#3~\vert~#4\Rightarrow#5}
\newcommand{\dadd}[2]{#1 + #2}
%% TODO third arg should be empty
\newcommand{\dehole}[3]{\leftidx{^{#3}}{\llparenthesiscolor\rrparenthesiscolor}{^{#1}_{#2}}}
%% TODO fourth arg should be empty
\newcommand{\dhole}[4]{\leftidx{^{#4}}{\llparenthesiscolor#1\rrparenthesiscolor}{^{#2}_{#3}}}
\newcommand{\dindet}[1]{\lceil#1\rceil}
%\newcommand{\dinj}[2]{\texttt{inj}_{#1}({#2})}
\newcommand{\dinL}[2]{\mathsf{inl}_{#1}(#2)}
\newcommand{\dinR}[2]{\mathsf{inr}_{#1}(#2)}
\newcommand{\dcase}[5]{\texttt{case}({#1},{#2}.{#3},{#4}.{#5})}
\newcommand{\dpair}[2]{(#1,#2)}
\newcommand{\dprj}[2]{\mathsf{prj}_{#1}(#2)}

\newcommand{\expandAna}[6]{#1 \vdash #2 \Leftarrow #3 \leadsto #4 : #5 \dashv #6}
\newcommand{\expandSyn}[5]{#1 \vdash #2 \Rightarrow #3 \leadsto #4 \dashv #5}
\newcommand{\hasType}[4]{#1; #2 \vdash #3 : #4}
\newcommand{\isValue}[1]{#1~\mathsf{val}}
\newcommand{\isGround}[1]{#1~\mathsf{ground}}
\newcommand{\isBoxedValue}[1]{#1~\mathsf{boxedval}}
\newcommand{\isIndet}[1]{#1~\mathsf{indet}}
\newcommand{\isFinal}[1]{#1~\mathsf{final}}
\newcommand{\isErr}[2]{#1 \vdash #2~\mathsf{err}}
%% \newcommand{\stepsTo}[2]{#1 \mapsto_{\Delta} #2}
%% TODO first arg should be empty
%% \newcommand{\stepsToD}[3]{#1 \vdash #2 \mapsto #3}
\newcommand{\stepsToD}[3]{#2 \mapsto #3}
\newcommand{\multiStepsTo}[2]{#1 \mapsto^* #2}

%% TODO if feeling obsessive, replace direct uses of \Delta
\newcommand{\hDelta}{\Delta}
\newcommand{\Dunion}[2]{#1 \cup #2}
\newcommand{\idof}[1]{\mathsf{id}(#1)}
\newcommand{\Dbinding}[3]{#1 :: #3[#2]}
\newcommand{\instantiate}[3]{\llbracket#1 / #2\rrbracket #3}

% Contextual dynamics
\newcommand{\evalctx}{\mathcal{E}}
\newcommand{\evalhole}{\circ}
\newcommand{\isevalctx}[1]{#1~\mathsf{evalCtx}}
%% TODO first arg should be empty
%% \newcommand{\reducesE}[3]{#1 \vdash #2 \longrightarrow #3}
\newcommand{\reducesE}[3]{#2 \longrightarrow #3}
\newcommand{\selectEvalCtxR}[2]{#1\{#2\}}
\newcommand{\selectEvalCtx}[3]{#1=\selectEvalCtxR{#2}{#3}}
\newcommand{\maybePremise}[1]{{\textcolor{red}[}#1{\textcolor{red}]}}

\newcommand{\inhole}[2]{\mathsf{inhole}(#1; #2)}

\newcommand{\DoSubst}[3]{[#1/#2]{#3}}


%\setlength{\abovecaptionskip}{4pt plus 3pt minus 2pt} % Chosen fairly arbitrarily
%\setlength{\belowcaptionskip}{-4pt plus 3pt minus 2pt} % Chosen fairly arbitrarily


\begin{document}

%% Title information
\title{Data Structures for Dictionaries in Proof Assistants}

%% Author information
%% Contents and number of authors suppressed with 'anonymous'.
%% Each author should be introduced by \author, followed by
%% \authornote (optional), \orcid (optional), \affiliation, and
%% \email.
%% An author may have multiple affiliations and/or emails; repeat the
%% appropriate command.
%% Many elements are not rendered, but should be provided for metadata
%% extraction tools.

% ACM member number: ?
% \author{?}
% \authornote{with author1 note}          %% \authornote is optional;
                                        %% can be repeated if necessary
% \orcid{nnnn-nnnn-nnnn-nnnn}             %% \orcid is optional
%\affiliation{
%  \position{Graduate student}
%  % \department{Department1}              %% \department is recommended
%  \institution{University of Chicago}            %% \institution is required
%  % \streetaddress{Street1 Address1}
%  % \city{City1}
%  % \state{State1}
%  % \postcode{Post-Code1}
%  % \country{Country1}
%}
%\email{nickmc@uchicago.edu}          %% \email is recommended

%% Paper note
%% The \thanks command may be used to create a "paper note" ---
%% similar to a title note or an author note, but not explicitly
%% associated with a particular element.  It will appear immediately
%% above the permission/copyright statement.
% \thanks{with paper note}                %% \thanks is optional
                                        %% can be repeated if necesary
                                        %% contents suppressed with 'anonymous'

TODO TODO TODO remaining info, e.g. ACM member number and author info or whatev

%% Abstract
%% Note: \begin{abstract}...\end{abstract} environment must come
%% before \maketitle command
% !TEX root = ?.tex

%% no citations in Abstract
%%
%% \cite{?}

\begin{abstract}
Dictionaries (a.k.a. finite maps) are a core tool in any programming environment.
Conventional languages implement dictionaries with hashtables or auto-balanced trees, but proof assistants
- which favor simplicity and provability over performance - generally use association lists
(with or without canonical ordering) or finite partial functions. These solutions are suitable for many
cases - however, each one has a drawback: unordered lists can contain duplicates or have arbitrary ordering,
canonically ordered lists may be invalid and so must be refined with a proof of validity, and equality of finite partial
functions is undecidable. Using a variant of delta-encoding, we develop a novel list-based solution that preserves
canonical ordering while ensuring that every type-correct list is semantically valid, eliminating the need for refinement
with a proof of validity. Our solution also establishes a one-to-one correspondence between dictionary terms and semantic
mappings. We demonstrate the practical importance of these properties when developing metatheory for environment-based
evaluation semantics, while acknowledging that drawbacks of our solution may make it inferior to one of the conventional
solutions in some cases where that conventional solution is viable. We prove the relevant metatheory in Agda.
%
% Our solution is not suitable for all key types, and is harder to destruct or
%iterate than other list-of-pair solutions, but is nonetheless better suited to many large, complex program
%foundations metatheories than the existing solutions.
%
% Although easy to define and reason about, this approach has the disadvantage that keys
%can be duplicated, leading to the fact that many distinct lists represent the same semantic mapping.
%This many-one relationship creates various difficulties when used in practice, such as when proving
%contraction and exchange for a type-checking judgment.
\end{abstract}

% Ordinary (value) types (ty-)
\newcommand{\ty}[0]{\tau} % Meta variable for types
\newcommand{\tyHref}[1]{\keyword{Href}\left<#1\right>} % Hole reference type, parameterized by the type within the hole
\newcommand{\tyNum}[0]{\keyword{Num}}
\newcommand{\tySum}[2]{#1 + #2}
\newcommand{\tyProd}[2]{#1 \times #2}

% Expression forms (e-)
\newcommand{\e}[0]{e} % Meta variable for expressions
\newcommand{\ePalLet}[3]{\keyword{let palette}\,#1\,\keyword{=}\,#2\,\keyword{in}\,#3}
\newcommand{\ePalApp}[2]{#1~#2}

% Palette types (pt-)
\newcommand{\pt}[0]{T} % Meta variable for palette types
\newcommand{\ptPal}[2]{\keyword{Palette}\,#1\,#2}
\newcommand{\ptArr}[2]{#1 \rightarrow #2}

% Palette expressions (p-)               % 
\newcommand{\p}{p} % Meta variable for   % 
                   % palette expressions % From Ravi's notes:
\newcommand{\pDef}[1]{#1}                %   pd
\newcommand{\pLam}[2]{\lambda#1.#2}      %   \x -> p
\newcommand{\pApp}[2]{#1~#2}             %   p e

% Palette definitions (pd-)
\newcommand{\pd}{D} % Meta variable for palette definitions
\newcommand{\pdBody}[7]{\{ #1, #2, #3, #4, #5, #6, #7 \}} % (the body of) a palette definition
\newcommand{\pdBodyTwoRows}[7]{\left\{ \begin{array}{l} #1, #2, #3, \\ #4, #5, #6, #7 \end{array} \right\}} % (the body of) a palette definition

% Palette typing context (ascribes a palette type to each palette variable)
\newcommand{\Ptc}[0]{\Delta}        % Meta variable
\newcommand{\PtcEmp}[0]{\cdot}      % Empty
\newcommand{\PtcBind}[3]{#1, #2:#3} % Binding

% Value typing context (ascribes a value type to each value variable)
\newcommand{\Ctx}{\Gamma}           % Meta variable
\newcommand{\CtxEmp}[0]{\cdot}      % Empty
\newcommand{\CtxBind}[3]{#1, #2:#3} % Binding


%% Keywords
%% comma separated list
% \keywords{keyword1, keyword2, keyword3}  %% \keywords is optional
% \keywords{..., ...}

%% \maketitle
%% Note: \maketitle command must come after title commands, author
%% commands, abstract environment, Computing Classification System
%% environment and commands, and keywords command.
\maketitle
\thispagestyle{empty} % suppresses the footer

\section{Introduction}
Conventionally, the design of data structures and algorithms is chiefly focused on performance.
In proof assistants, however, much of the logic is never actually executed - such logic can afford
to be arbitarily unperformant. This observation suggests a new tack in data structure design for
the domain of proof assistants - long-solved problems are worth reconsidering from a perspective
that prioritizes simplicity, elegance, or useful mathematical properties over performance.

In this paper we reconsider the long-solved problem of representing dictionaries (a.k.a. finite maps).
Dictionaries are a key
mechanism for data storage and retrieval in almost any programming paradigm - although perhaps less
ubiquitous in proof languages, they still serve critical roles such as representing contexts and
environments in type checking and evaluation semantics (resp.). Thus, development of new approaches to
dictionaries that are better suited to the unique needs of proof assistants can be highly profitable.

In conventional languages,
dictionaries are usually represented either with auto-balanced binary search trees or hashtables, which
are performant but complicated and hard to prove correct. In proof assistants, it's more typical to
use a list-of-pairs (i.e. associative array), which has $O(n)$ lookup performance but is very simple.
Since the typical approach is already a paragon of simplicity at the expense of performance, what's left
to reconsider? Its shortcoming is its mathematical awkwardness; in most use cases, dictionaries are
expected to be canonically ordered (or order-agnostic) and have at most one mapping per key,
properties which are upheld by BSTs and hashtables
but which are abandoned by the ordering- and duplication-sensitive list-of-pairs. In particular,
the \emph{structural properties} \emph{contraction} and \emph{exchange} of a type checking judgment (or
other judgments parameterized by some sort of context) come "for free" if the context is represented by
a set-like data structure that is agnostic to insertion order and which does not allow duplicate keys.
This is a substantial benefit compared to deriving a separate proof of contraction and exchange for each
judgment that takes a context parameter.

To address this issue we introduce three (TODO ????) novel dictionary representations that are agnostic
to insertion order and duplication - implicitly finite function, explicitly finite functions, and an
alternative list-of-pairs approach that uses a variant of delta-encoding to maintain keys in sorted order
while preventing duplication. As such, using one of these representations grants \emph{contraction} and
\emph{exchange} for free, along with other benefits of their ordering- and duplication-agnostic properties
(the delta-encoding approach, in particular, has the additional property that physical equality is
equivalent to semantic equivalence). However, they are not panaceas - the typical list-of-pair data structure
is much easier to destruct and iterate than the new approaches, and the delta-encoding approach is limited
to key types that are easily mapped to and from the natural numbers. Altogether, we're left with a rich
variety of options for representing dictionaries, each well-suited to some use-cases but not others. This
situation is not so dissimilar to the one familiar to conventional programmers, who generally face the choice
of either BSTs or hashtables - further refinement to the point of making such an easy choice between only two
options is left to future work.

TODO TODO TODO - REFERENCES

\clearpage
%\bibliography{references,all.short}
\bibliography{references}

% \clearpage
% \appendix
% \input{implementation-appendix}

\end{document}
