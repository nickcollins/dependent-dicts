\PassOptionsToPackage{svgnames,dvipsnames,svgnames}{xcolor}

\documentclass[nonacm]{acmart}
\settopmatter{printfolios=true,printccs=false,printacmref=false}

%% Conference information
%% Supplied to authors by publisher for camera-ready submission;
%% use defaults for review submission.
%% ?

%% Copyright information
%% Supplied to authors (based on authors' rights management selection;
%% see authors.acm.org) by publisher for camera-ready submission;
%% use 'none' for review submission.
\setcopyright{none}
%\setcopyright{acmcopyright}
%\setcopyright{acmlicensed}
%\setcopyright{rightsretained}
%\copyrightyear{2019}           %% If different from \acmYear

%% Bibliography style
\bibliographystyle{ACM-Reference-Format}
%% Citation style
%\citestyle{acmauthoryear}  %% For author/year citations
%\citestyle{acmnumeric}     %% For numeric citations
%\setcitestyle{nosort}      %% With 'acmnumeric', to disable automatic
                            %% sorting of references within a single citation;
                            %% e.g., \cite{Smith99,Carpenter05,Baker12}
                            %% rendered as [14,5,2] rather than [2,5,14].
%\setcitesyle{nocompress}   %% With 'acmnumeric', to disable automatic
                            %% compression of sequential references within a
                            %% single citation;
                            %% e.g., \cite{Baker12,Baker14,Baker16}
                            %% rendered as [2,3,4] rather than [2-4].


%% Some recommended packages.
\usepackage{booktabs}   %% For formal tables:
                        %% http://ctan.org/pkg/booktabs
\usepackage{subcaption} %% For complex figures with subfigures/subcaptions
                        %% http://ctan.org/pkg/subcaption

%% Cyrus packages
\usepackage{microtype}
\usepackage{mdframed}
\usepackage{colortab}
\usepackage{mathpartir}
\usepackage{enumitem}
\usepackage{bbm}
\usepackage{stmaryrd}
\usepackage{mathtools}
\usepackage{leftidx}
\usepackage{todonotes}
\usepackage{xspace}
\usepackage{wrapfig}

\usepackage{listings}%
\lstloadlanguages{ML}
\lstset{tabsize=2, 
basicstyle=\footnotesize\ttfamily, 
% keywordstyle=\sffamily,
commentstyle=\itshape\ttfamily\color{gray}, 
stringstyle=\ttfamily\color{purple},
mathescape=false,escapechar=\#,
numbers=left, numberstyle=\scriptsize\color{gray}\ttfamily, language=ML, showspaces=false,showstringspaces=false,xleftmargin=10pt, 
morekeywords={string, float, int},
classoffset=0,belowskip=\smallskipamount, aboveskip=\smallskipamount,
moredelim=**[is][\color{red}]{SSTR}{ESTR}
}
\newcommand{\li}[1]{\lstinline[basicstyle=\ttfamily\fontsize{9pt}{1em}\selectfont]{#1}}
\newcommand{\lismall}[1]{\lstinline[basicstyle=\ttfamily\fontsize{9pt}{1em}\selectfont]{#1}}

%% Joshua Dunfield macros
\def\OPTIONConf{1}%
\usepackage{joshuadunfield}

%% Can remove this eventually
\usepackage{blindtext}

\usepackage{enumitem}

% \newtheorem{theorem}{Theorem}[chapter]
% \newtheorem{lemma}[theorem]{Lemma}
% \newtheorem{corollary}[theorem]{Corollary}
% \newtheorem{definition}[theorem]{Definition}
% \newtheorem{assumption}[theorem]{Assumption}
% \newtheorem{condition}[theorem]{Condition}

\newtheoremstyle{slplain}% name
  {.15\baselineskip\@plus.1\baselineskip\@minus.1\baselineskip}% Space above
  {.15\baselineskip\@plus.1\baselineskip\@minus.1\baselineskip}% Space below
  {\slshape}% Body font
  {\parindent}%Indent amount (empty = no indent, \parindent = para indent)
  {\bfseries}%  Thm head font
  {.}%       Punctuation after thm head
  { }%      Space after thm head: " " = normal interword space;
        %       \newline = linebreak
  {}%       Thm head spec
\theoremstyle{slplain}
\newtheorem{thm}{Theorem}  % Numbered with the equation counter
\numberwithin{thm}{section}
\newtheorem{defn}[thm]{Definition}
\newtheorem{lem}[thm]{Lemma}
\newtheorem{prop}[thm]{Proposition}
% \newtheorem{cor}[section]{Corollary}     
% \newtheorem{lem}[section]{Lemma}         
% \newtheorem{prop}[section]{Proposition}  

% \setlength{\abovedisplayskip}{0pt}
% \setlength{\belowdisplayskip}{0pt}
% \setlength{\abovedisplayshortskip}{0pt}
% \setlength{\belowdisplayshortskip}{0pt}

\fancyfoot{} % suppresses the footer (also need \thispagestyle{empty} after \maketitle below)

% !TEX root = main.tex

\newcommand{\mynote}[3]{\textcolor{#3}{\textsf{{#2}}}}
\newcommand{\rkc}[1]{\mynote{rkc}{#1}{blue}}
\newcommand{\nick}[1]{\mynote{cy}{#1}{purple}}

\newcommand{\cvert}{{\,{\vert}\,}}

%% https://tex.stackexchange.com/questions/9796/how-to-add-todo-notes
\newcommand{\rkcTodo}[1]{\todo[linecolor=blue,backgroundcolor=blue!25,bordercolor=blue]{#1}}

\newcommand{\mattTodo}[1]{\todo[linecolor=green,backgroundcolor=green!2,bordercolor=green]{\tiny\textit{#1}}}
\newcommand{\mattOmit}[1]{\colorbox{yellow}{(Matt omitted stuff here)}}

\def\parahead#1{\paragraph{\textbf{#1.}}}
%% \def\paraheadNoDot#1{\paragraph{{\textbf{#1}}}}
\def\subparahead#1{\paragraph{\textit{#1.}}}
%% \def\paraheadindent#1{\paragraph{}\textit{#1.}}
%% \def\paraheadindentnodot#1{\paragraph{}\textit{#1}}

\newcommand{\ie}{i.e.}
\newcommand{\eg}{e.g.}
% \newcommand{\etc}{{\emph{etc.}}}
% \newcommand{\cf}{{\emph{cf.}}}
% \newcommand{\etal}{{\emph{et al.}}}

%% \newcommand{\hazel}{\ensuremath{\textsc{Hazel}}}
%% \newcommand{\sns}{\ensuremath{\textsc{Sketch-n-Sketch}}}
%% \newcommand{\deuce}{\ensuremath{\textsc{Deuce}}}
\newcommand{\Elm}{\ensuremath{\textsf{Elm}}}
\newcommand{\sns}{\ensuremath{\textrm{Sketch-n-Sketch}}}
\newcommand{\deuce}{\ensuremath{\textrm{Deuce}}}

\newcommand{\sectionDescription}[1]{\section{#1}}
\newcommand{\subsectionDescription}[1]{\subsection{#1}}
\newcommand{\subsubsectionDescription}[1]{\subsubsection{#1}}
%% \newcommand{\subsectionDescription}[1]{\subsection*{#1}}
\newcommand{\suppMaterials}{the Supplementary Materials}

\newcommand{\defeq}{\overset{\textrm{def}}{=}}

\newcommand{\eap}{action suggestion panel\xspace}
\newcommand{\Eap}{Action suggestion panel\xspace}

\newcommand{\myfootnote}[1]{\footnote{ #1}}

\def\sectionautorefname{Section}
\def\subsectionautorefname{Section}
\def\subsubsectionautorefname{Section}

\newcommand{\code}[1]{\lstinline{#1}}
\newcommand{\str}[1]
  {``#1''}

% Make italic?
%\newcommand{\Property}[1]{\emph{#1}}
\newcommand{\Property}[1]{\textrm{#1}}

% Calling out Cyrus's favorite verb, 'to be' ;)
\newcommand{\IS}{\colorbox{red}{is}\xspace}

\newcommand{\codeSize}
  %% {\footnotesize}
  {\small}

%\newcommand{\JoinTypes}[2]{\textsf{join}~~#1~~#2}
\newcommand{\JoinTypes}[2]{\textsf{join}(#1,#2)}

%% \newtheorem{theorem}{Theorem}[section] % 2.1, 2.2, etc.
\newtheorem{theorem}{Theorem}             % 1, 2, etc.
\newtheorem{remark}[theorem]{Remark}

%%%%%%%%%%%%%%%%%%%%%%%%%%%%%%%%%%%%%%%%%%%%%%%%%%%%%%%%%%%%%%%%%%%%%%%%%%%%%%%%
%% Spacing

\newcommand{\sep}{\hspace{0.06in}}
\newcommand{\sepPremise}{\hspace{0.20in}}
\newcommand{\hsepRule}{\hspace{0.20in}}
\newcommand{\vsepRuleHeight}{0.08in}
\newcommand{\vsepRule}{\vspace{\vsepRuleHeight}}
\newcommand{\miniSepOne}{\hspace{0.01in}}
\newcommand{\miniSepTwo}{\hspace{0.02in}}
\newcommand{\miniSepThree}{\hspace{0.03in}}
\newcommand{\miniSepFour}{\hspace{0.04in}}
\newcommand{\miniSepFive}{\hspace{0.05in}}
\newcommand{\breakAndIndent}
  {\mbox{}

   %% \hspace{0.15in}
   \hspace{0.00in}
  }

\newcommand{\justIndent}
  %% {\hspace{0.15in}
  {\hspace{0.00in}
  }

%%%%%%%%%%%%%%%%%%%%%%%%%%%%%%%%%%%%%%%%%%%%%%%%%%%%%%%%%%%%%%%%%%%%%%%%%%%%%%%%

% \lstset{
% %mathescape=true,basicstyle=\fontsize{8}{9}\ttfamily,
% literate={=>}{$\Rightarrow$}2
%          {<=}{$\leq$}2
%          {->}{${\rightarrow}$}1
%          {\\\\=}{\color{red}{$\lambda$}}2
%          {\\\\}{$\lambda$}2
%          {**}{$\times$}2
%          {*.}{${\color{blue}{\texttt{*.}}}$}2
%          {+.}{${\color{blue}{\texttt{+.}}}$}2
%          {<}{${\color{green}{\lhd}}$}1
%          {>?}{${\color{green}{\rhd}}$?}2
%          {<<}{${\color{green}{\blacktriangleleft}}$}1
%          {>>?}{${\color{green}{\blacktriangleright}}$?}2
%          {\{}{${\color{blue}{\{}}$}1
%          {\}}{${\color{blue}{\}}}$}1
%          {[}{${\color{purple}{[}}$}1
%          {]}{${\color{purple}{]}}$}1
%          {(}{${\color{darkgray}{\texttt{(}}}$}1
%          {)}{${\color{darkgray}{\texttt{)}}}$}1
%          {]]}{${\color{gray}{\big(}}$}1
%          {]]}{${\color{gray}{\big)}}$}1
% }

%%%%%%%%%%%%%%%%%%%%%%%%%%%%%%%%%%%%%%%%%%%%%%%%%%%%%%%%%%%%%%%%%%%%%%%%%%%%%%%%

\newcommand{\li}[1]{\lstinline[basicstyle=\ttfamily\fontsize{9pt}{1em}\selectfont]{#1}}
\newcommand{\lismall}[1]{\lstinline[basicstyle=\ttfamily\fontsize{9pt}{1em}\selectfont]{#1}}
\newcommand{\mapP}{\{0 \rightarrow \text{"a"}, 1 \rightarrow \text{"b"}, 3 \rightarrow \text{"c"}, 6 \rightarrow \text{"d"}, 10 \rightarrow \text{"e"}\}}
\newcommand{\dictP}{[(0, "a"), (0, "b"), (1, "c"), (2, "d"), (3, "e")]}

%%%%%%%%%%%%%%%%%%%%%%%%%%%%%%%%%%%%%%%%%%%%%%%%%%%%%%%%%%%%%%%%%%%%%%%%%%%%%%%%

%% \newcommand{\sal}{simple association list}
\newcommand{\sal}{association list}
\newcommand{\Sal}{Association List}

%% \newcommand{\cal}{canonicalized association list}
\newcommand{\cal}{canonical association list}
\newcommand{\Cal}{Canonical Association List}
\newcommand{\Cals}{Canonical association lists}

%% \newcommand{\fpf}{finite partial function}
\newcommand{\fpf}{partial function}
\newcommand{\Fpf}{Partial Function}
\newcommand{\Fpfs}{Partial functions}

\newcommand{\dd}{delta dictionary}
\newcommand{\dds}{delta dictionaries}
\newcommand{\Dd}{Delta Dictionary}
\newcommand{\Ddls}{Delta dictionaries}

%%%%%%%%%%%%%%%%%%%%%%%%%%%%%%%%%%%%%%%%%%%%%%%%%%%%%%%%%%%%%%%%%%%%%%%%%%%%%%%%

\newcommand{\SemTot}{Semantic Totality}
%% \newcommand{\SemInj}{Semantic Injectivity}
\newcommand{\SemInj}{Extensionality}
\newcommand{\EqDec}{Decidable Equality}
%% \newcommand{\EzDstr}{Ease of Destruction}
\newcommand{\EzDstr}{Easy Destructibility}

%% tweak and remove indirection macros later, once terminology settles
\newcommand{\Total}{\SemTot}
\newcommand{\Extensional}{\SemInj}
\newcommand{\DecidableEq}{\EqDec}
\newcommand{\Destructible}{\EzDstr}

\newcommand{\total}{Total}
\newcommand{\extensional}{Extensional}
\newcommand{\decidable}{Decid. Eq.}
\newcommand{\destructible}{Destructible}

\newcommand{\semanticallyTotal}{Semantically Total}
\newcommand{\destructed}{Destructed}

%%%%%%%%%%%%%%%%%%%%%%%%%%%%%%%%%%%%%%%%%%%%%%%%%%%%%%%%%%%%%%%%%%%%%%%%%%%%%%%%

% for use in alltt
\newcommand{\altFAll}{\ensuremath{\forall}}
\newcommand{\altRArr}{\ensuremath{\rightarrow}}
\newcommand{\altRARR}{\ensuremath{\Rightarrow}}
\newcommand{\altVdash}{\ensuremath{\vdash}}
\newcommand{\altTimes}{\ensuremath{\times}}
\newcommand{\altAnd}{\ensuremath{\land}}
\newcommand{\altOr}{\ensuremath{\lor}}
\newcommand{\altNE}{\ensuremath{\ne}}
\newcommand{\altEmpty}{\ensuremath{\varnothing}}
\newcommand{\altIn}{\ensuremath{\in}}
\newcommand{\altNIn}{\ensuremath{\notin}}
\newcommand{\altSum}{\ensuremath{\sum}}
\newcommand{\altLAng}{\ensuremath{\langle}}
\newcommand{\altRAng}{\ensuremath{\rangle}}
\newcommand{\altLamb}{\ensuremath{\lambda}}
\newcommand{\altGam}{\ensuremath{\Gamma}}
\newcommand{\altCirc}{\ensuremath{\circ}}
\newcommand{\altCdot}{\ensuremath{\cdot}}
\newcommand{\altBot}{\ensuremath{\bot}}

\input{macros}

%\setlength{\abovecaptionskip}{4pt plus 3pt minus 2pt} % Chosen fairly arbitrarily
%\setlength{\belowcaptionskip}{-4pt plus 3pt minus 2pt} % Chosen fairly arbitrarily


\begin{document}

%% Title information
\title{Data Structures for Dictionaries in Proof Assistants}

%% Author information
%% Contents and number of authors suppressed with 'anonymous'.
%% Each author should be introduced by \author, followed by
%% \authornote (optional), \orcid (optional), \affiliation, and
%% \email.
%% An author may have multiple affiliations and/or emails; repeat the
%% appropriate command.
%% Many elements are not rendered, but should be provided for metadata
%% extraction tools.

% ACM member number: ?
% \author{?}
% \authornote{with author1 note}          %% \authornote is optional;
                                        %% can be repeated if necessary
% \orcid{nnnn-nnnn-nnnn-nnnn}             %% \orcid is optional
%\affiliation{
%  \position{Graduate student}
%  % \department{Department1}              %% \department is recommended
%  \institution{University of Chicago}            %% \institution is required
%  % \streetaddress{Street1 Address1}
%  % \city{City1}
%  % \state{State1}
%  % \postcode{Post-Code1}
%  % \country{Country1}
%}
%\email{nickmc@uchicago.edu}          %% \email is recommended

%% Paper note
%% The \thanks command may be used to create a "paper note" ---
%% similar to a title note or an author note, but not explicitly
%% associated with a particular element.  It will appear immediately
%% above the permission/copyright statement.
% \thanks{with paper note}                %% \thanks is optional
                                        %% can be repeated if necesary
                                        %% contents suppressed with 'anonymous'

TODO TODO TODO remaining info, e.g. ACM member number and author info or whatev

%% Abstract
%% Note: \begin{abstract}...\end{abstract} environment must come
%% before \maketitle command
% !TEX root = ?.tex

%% no citations in Abstract
%%
%% \cite{?}

\begin{abstract}

Dictionaries, or finite maps, are a core data structure in any programming language.
%
General-purpose languages implement dictionaries with hash tables or balanced trees, but proof assistants---which favor simplicity and provability over performance---generally employ association lists or finite partial functions.
%
Though suitable for many uses, each solution has drawbacks:
%
association lists allow arbitrary order and may contain duplicates (unless refined with an explicit proof about validity), and
%
partial functions cannot be compared for equality
%
nor easily destructed.

We develop a novel list-based representation, called \emph{delta dictionaries}, that simultaneously achieves benefits of the conventional representations.
%
Delta dictionaries are \emph{total} (every type-correct list is semantically valid),
%
\emph{extensional} (there is a one-to-one correspondence between lists and semantic mappings), and
%
\emph{destructible} (the internal representation can be inspected as needed).
%
We present the implementation of delta dictionaries and relevant metatheory---formalized in Agda---and we discuss when delta dictionaries may or may not be preferable to conventional solutions.

%% Using a variant of delta-encoding, we develop a novel list-based solution that preserves
%% canonical ordering while ensuring that every type-correct list is semantically valid, eliminating the need for refinement
%% with a proof of validity. Our solution also establishes a one-to-one correspondence between dictionary terms and semantic
%% mappings. We demonstrate the practical importance of these properties when developing metatheory for environment-based
%% evaluation semantics, while acknowledging that drawbacks of our solution may make it inferior to one of the conventional
%% solutions in some cases where that conventional solution is viable. We prove the relevant metatheory in Agda.

%
% Our solution is not suitable for all key types, and is harder to destruct or
%iterate than other list-of-pair solutions, but is nonetheless better suited to many large, complex program
%foundations metatheories than the existing solutions.
%
% Although easy to define and reason about, this approach has the disadvantage that keys
%can be duplicated, leading to the fact that many distinct lists represent the same semantic mapping.
%This many-one relationship creates various difficulties when used in practice, such as when proving
%contraction and exchange for a type-checking judgment.

\end{abstract}

\input{defs}

%% Keywords
%% comma separated list
% \keywords{keyword1, keyword2, keyword3}  %% \keywords is optional
% \keywords{..., ...}

%% \maketitle
%% Note: \maketitle command must come after title commands, author
%% commands, abstract environment, Computing Classification System
%% environment and commands, and keywords command.
\maketitle
\thispagestyle{empty} % suppresses the footer

\section{Introduction}
Conventionally, the design of data structures and algorithms is chiefly focused on performance.
In proof assistants, however, much of the logic is never actually executed - such logic can afford
to be arbitarily unperformant. This observation suggests a new tack in data structure design for
the domain of proof assistants - long-solved problems are worth reconsidering from a perspective
that prioritizes simplicity, elegance, or useful mathematical properties over performance.

In this paper we reconsider the long-solved problem of representing dictionaries (a.k.a. finite maps).
Dictionaries are a key
mechanism for data storage and retrieval in almost any programming paradigm - although perhaps less
ubiquitous in proof languages, they still serve critical roles such as representing contexts and
environments in type checking and evaluation semantics (resp.). Thus, development of new approaches to
dictionaries that are better suited to the unique needs of proof assistants can be highly profitable.

In conventional languages,
dictionaries are usually represented either with auto-balanced binary search trees or hashtables, which
are performant but complicated and hard to prove correct. In proof assistants, it's more typical to
use a list-of-pairs (i.e. associative array), which has $O(n)$ lookup performance but is very simple.
Since the typical approach is already a paragon of simplicity at the expense of performance, what's left
to reconsider? Its shortcoming is its mathematical awkwardness; in most use cases, dictionaries are
expected to be canonically ordered (or order-agnostic) and have at most one mapping per key,
properties which are upheld by BSTs and hashtables
but which are abandoned by the ordering- and duplication-sensitive list-of-pairs. In particular,
the \emph{structural properties} \emph{contraction} and \emph{exchange} of a type checking judgment (or
other judgments parameterized by some sort of context) come almost "for free" if the context is represented by
a set-like data structure that is agnostic to insertion order and which does not allow duplicate keys.
This is a substantial benefit compared to deriving a separate proof of contraction and exchange for each
judgment that takes a context parameter.

To address this issue we introduce three (TODO ????) novel dictionary representations that are agnostic
to insertion order and duplication - implicitly finite function, explicitly finite functions, and an
alternative list-of-pairs approach that uses a variant of delta-encoding to maintain keys in sorted order
while preventing duplication. As such, using one of these representations means \emph{contraction} and
\emph{exchange} need be proven only once, for the dictionary itself, rather than a separate proof for each
context-sensitive judgment. There are additional benefits of the ordering- and duplication-agnostic properties
- the delta-encoding approach, in particular, has the additional property that physical equality is
equivalent to semantic equivalence. However, they are not panaceas - the typical list-of-pair data structure
is much easier to destruct and iterate than the new approaches, and the delta-encoding approach is limited
to key types that are easily mapped to and from the natural numbers. Altogether, we're left with a rich
variety of options for representing dictionaries, each well-suited to some use-cases but not others. This
situation is not so dissimilar to the one familiar to conventional programmers, who generally face the choice
of either BSTs or hashtables - further refinement to the point of making such an easy choice between only two
options is left to future work.

TODO TODO TODO - REFERENCES
\\
the second answer on this stack overflow suggests the finite partial function approach:
\\
https://stackoverflow.com/questions/47362451/creating-a-dictionary-map-in-coq
\\
oh hey appears that this is in software foundations too
\\
https://6826.csail.mit.edu/2019/lf/Maps.html
\\
Coq has list-based maps that retain sorted order, but not every list data is a valid mapping
\\
https://coq.inria.fr/library/Coq.FSets.FMapList.html
\\
Coq also has FMapPositive, which is a tree based on the binary representation
\\
This one seems to attach a canonicity proof to an ordinary list:
\\
http://www.cs.bc.edu/~tassarot/papers/iris-refinement/coqdoc/iris.prelude.natmap.html

\clearpage
%\bibliography{references,all.short}
\bibliography{references}

% \clearpage
% \appendix
% \input{implementation-appendix}

\end{document}
